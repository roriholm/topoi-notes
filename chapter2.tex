\chapter{What Categories Are}

\section{Functions as Sets}
    We are only going to wallow in the swamp of Set Theory for a brief moment longer.
    Given a function $f: A \to B$ we can derive some related sets.
    \begin{defi}
        The \emph{relation}
        $$\hat{f} := \set{ \langle a, f(a) \rangle \st a \in A}.$$ 
    \end{defi}
    \begin{defi}
        The \emph{image set} 
        $$f(A) := \set{b \in B \st  b = f(a) \text{ for some } a \in A}.$$ 
        Equivalently
        $$f(A) := \set{b \st \langle a, b \rangle \in \hat{f} \text { for some } a \in A}.$$
    \end{defi}

\section{Composition}

    The power of composition is that it can't be resisted.
    Say I hand you an $f : A \to B$ and a $G : B \to C$. 
    Then there is a clear procedure for getting from $A$ to $C$.
    \begin{enumerate}[(1)]
        \item Take your $a \in A$.
        \item Apply $f$ to $a$, yielding $f(a)$ which is some $b \in B$.
        \item Apply $g$ to $b$, yielding $g(b)$, which is some $c \in C$
    \end{enumerate}
    This procedure yields the
    \begin{defi}
        \emph{Function composition}
        $$g\circ f.$$
    \end{defi}

    \begin{fact}
        Functions are associative, or
        $$(f \o g) \o h = f \o (g \o h).$$
    \end{fact}
    \begin{exercise}
        Convince yourself that functions are associative.
    \end{exercise}

    \subsection{Identities}
    Given any set $B$ there is an important function that comes for free. We call it $1_B : B \to B$ and it's given by
    $$1_B(b) = b$$
    for any $b \in B$.

    \begin{fact}
        Identities are absorbed left and right. In other words, if we have $f : A \to B$ and a $G : B \to C$, then
        $$1_B \o f = f$$
        and
        $$g \o 1_B = g.$$
        That gives us our first commutative diagram, for which any path is equivalent:
        % https://q.uiver.app/?q=WzAsNCxbMCwxLCJBIl0sWzIsMiwiQiJdLFsyLDAsIkIiXSxbNCwxLCJDIl0sWzAsMiwiZiJdLFswLDEsImYiLDJdLFsxLDMsImciLDJdLFsyLDMsImciXSxbMiwxLCIxX0IiLDEseyJzdHlsZSI6eyJ0YWlsIjp7Im5hbWUiOiJhcnJvd2hlYWQifX19XV0=
        \[\begin{tikzcd}
            && {B} \\
            {A} &&&& {C} \\
            && {B}
            \arrow["{f}", from=2-1, to=1-3]
            \arrow["{f}"', from=2-1, to=3-3]
            \arrow["{g}"', from=3-3, to=2-5]
            \arrow["{g}", from=1-3, to=2-5]
            \arrow["{1_B}" description, from=1-3, to=3-3, tail reversed]
        \end{tikzcd}\]
    \end{fact}

\section{Category Axioms}
    We are ready for the abstract view of the above.

    \begin{defi}
        A \emph{category} has 
        \begin{enumerate}
            \item a collection of objects:
            $$a,b,c.$$
            \item a collection of arrows, each with specific domain and codomain:
            $$f : a \to b,$$
            $$g : b \to c.$$
            \item an associative composition operation that yields a unique arrow `skipping' aligned domains and codomains:
            $$g\o f : a \to c.$$
            \item an identity arrow for each object:
            $$f \circ 1_A = f$$
            $$1_B \circ f = f$$
            $$g \circ 1_B = g$$
            $$1_C \circ g = g$$
        \end{enumerate}
    \end{defi}

    For now we will assume that the collection of arrows between any two objects is `only' a set's worth. In other words we will only deal with
    \emph{locally small} categories for now. (Riehl Page 7)

    \begin{fact}
        Diagrams are a convenient way to present categories:
        % https://q.uiver.app/?q=WzAsMyxbMCwwLCJBIl0sWzIsMCwiQiJdLFs0LDAsIkMiXSxbMCwxLCJmIiwyXSxbMSwyLCJnIiwyXV0=
        \[\begin{tikzcd}
            {A} && {B} && {C}
            \arrow["{f}"', from=1-1, to=1-3]
            \arrow["{g}"', from=1-3, to=1-5]
        \end{tikzcd}\]
        But you must keep in mind there are some implicit things not being shown here. The above is a compact verion of the following:
        % https://q.uiver.app/?q=WzAsNixbMCwyLCJBIl0sWzIsMiwiQiJdLFs0LDIsIkMiXSxbMCwwLCJBIl0sWzIsMCwiQiJdLFs0LDAsIkMiXSxbMCwxLCJmIiwyXSxbMSwyLCJnIiwyXSxbMyw0LCJmIiwyXSxbNCw1LCJnIiwyXSxbMywwLCIxX0EiLDEseyJzdHlsZSI6eyJ0YWlsIjp7Im5hbWUiOiJhcnJvd2hlYWQifX19XSxbNCwxLCIxX0IiLDEseyJzdHlsZSI6eyJ0YWlsIjp7Im5hbWUiOiJhcnJvd2hlYWQifX19XSxbNSwyLCIxX0MiLDEseyJzdHlsZSI6eyJ0YWlsIjp7Im5hbWUiOiJhcnJvd2hlYWQifX19XSxbMCwyLCJnIFxcbyBmIiwxLHsiY3VydmUiOjN9XSxbMyw1LCJnIFxcbyBmIiwxLHsiY3VydmUiOi0zfV1d
        \[\begin{tikzcd}
            {A} && {B} && {C} \\
            \\
            {A} && {B} && {C}
            \arrow["{f}"', from=3-1, to=3-3]
            \arrow["{g}"', from=3-3, to=3-5]
            \arrow["{f}"', from=1-1, to=1-3]
            \arrow["{g}"', from=1-3, to=1-5]
            \arrow["{1_A}" description, from=1-1, to=3-1, tail reversed]
            \arrow["{1_B}" description, from=1-3, to=3-3, tail reversed]
            \arrow["{1_C}" description, from=1-5, to=3-5, tail reversed]
            \arrow["{g \o f}" description, from=3-1, to=3-5, curve={height=18pt}]
            \arrow["{g \o f}" description, from=1-1, to=1-5, curve={height=-18pt}]
        \end{tikzcd}\]
    \end{fact}

    \begin{fact}
        Saying a diagram `commutes' is a convenient way to present equivalent compositions of arrows. For example, if 
        % https://q.uiver.app/?q=WzAsMyxbMCwwLCJBIl0sWzIsMCwiQiJdLFs0LDAsIkMiXSxbMCwxLCJmIiwyXSxbMSwyLCJnIiwyXSxbMCwyLCJoIiwwLHsiY3VydmUiOi0yfV1d
        \[\begin{tikzcd}
            {A} && {B} && {C}
            \arrow["{f}"', from=1-1, to=1-3]
            \arrow["{g}"', from=1-3, to=1-5]
            \arrow["{h}", from=1-1, to=1-5, curve={height=-12pt}]
        \end{tikzcd}\]
        commutes then we are saying that 
        $$g \o f = h.$$
    \end{fact}

    \begin{defi}
        Given two objects $a, b$ in a category $C$, we define the \emph{hom-set}
        $$C(a,b) \quad \text{or} \quad \hom(a,b)$$
        as the set of all arrows from $a$ to $b$.
    \end{defi}

\section{Comma Categories}

    \begin{defi}
        The comma-category $C\downarrow a$ is formed from any category $C$ and any $C$-object $a$.
        Its objects are all the $C$-arrows with codomain $a$. (IE, $f_1: c_1 \to a$ and $f_2: c_2 \to a$).
        Its arrows are all $C$-arrows between the objects' domains, that commute with the `object arrows'.
        (IE, $g : c_1 \to c_2$ so that $f_1 = f_2 \circ g$).
        So $C \downarrow a$ looks like this:
        \[\begin{tikzcd}
            {f_1} && {f_2}
            \arrow["{g}", from=1-1, to=1-3]
        \end{tikzcd}\]
        and indicates that this diagram commutes in the original category:
        \[\begin{tikzcd}
            {c_1} && {c_2} \\
            \\
            & {a}
            \arrow["{f_1}"', from=1-1, to=3-2]
            \arrow["{f_2}", from=1-3, to=3-2]
            \arrow["{g}", from=1-1, to=1-3]
        \end{tikzcd}\]

        (TODO: Verify category axioms)
    \end{defi}

    \begin{example}
        Take $C$ to be the preorder on natural numbers, and $a$ to be a given number.
        For example, let $a = 3$.

        \[\begin{tikzcd}
            {1} & {2} & {\textbf{3}} & {4} & {5} & {\ldots}
            \arrow[from=1-1, to=1-2]
            \arrow[from=1-2, to=1-3]
            \arrow[from=1-3, to=1-4]
            \arrow[from=1-4, to=1-5]
            \arrow[from=1-5, to=1-6]
        \end{tikzcd}\]

        Objects in $C \downarrow 3$ are statements of `$n \leq 4$' relationships.
        Arrows in $C \downarrow 3$ from `$m \leq 3$' to `$n\leq 3$' are `$m\leq n$'.

        \[\begin{tikzcd}
            {(1 \leq 3)} && {(2 \leq 3)} && {(3 \leq 3)}
            \arrow["{1 \leq 2}"', from=1-1, to=1-3]
            \arrow["{1 \leq 3}", from=1-1, to=1-5, curve={height=-18pt}]
            \arrow["{2 \leq 3}"', from=1-3, to=1-5]
        \end{tikzcd}\]
    \end{example}


    \begin{example}
        Recall $\textbf{Matr(k)}$ has the natural numbers $\N$ as objects, and $(n\times m)$ matrices as arrows from $m \to n$.

        Then $\textbf{Matr(k)} \downarrow 3$ has as objects all $3\times n$ matrices where $n \in \N$.

        Then if object $A$ is a $3\times m$ matrix, and object $B$ is a $3\times n$ matrix,
        then an arrow $C : A \to B$ is an $n \times m$ matrix such that $A=BC$.

        Thus the situation in $\textbf{Matr(k)} \downarrow 3$ is
        \[\begin{tikzcd}
            {A_{3\times m}} && {B_{3\times n}}
            \arrow["{C_{n\times m}}", from=1-1, to=1-3]
        \end{tikzcd}\]

        indicating that this diagram commutes in $\textbf{Matr(k)}$:
        \[\begin{tikzcd}
            {m} && {n} \\
            \\
            & {3}
            \arrow["{A_{3\times m}}"', from=1-1, to=3-2]
            \arrow["{B_{3\times n}}", from=1-3, to=3-2]
            \arrow["{C_{n\times m}}", from=1-1, to=1-3]
        \end{tikzcd}\]
    \end{example}
