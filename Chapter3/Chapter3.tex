\documentclass[12pt]{article}
\usepackage{../common}
\begin{document}

\chapter{3}{Arrows Instead of Epsilon}

\section{Monic arrows}
    Monic arrows are an abstraction of injective functions.
    \begin{definition}
    An arrow $f:a\to b$ in a category $C$ is \emph{monic} if for any $g_1, g_2$ with codomain $a$, the implication
    $$f \circ g_1 = f \circ g_2 \implies g_1 = g_2$$
    holds. Or, if the diagram
    \[\begin{tikzcd}
        {c} && {a} && {b}
        \arrow["{f}", from=1-3, to=1-5]
        \arrow["{g_1}", from=1-1, to=1-3, curve={height=-6pt}]
        \arrow["{g_2}"', from=1-1, to=1-3, curve={height=6pt}]
    \end{tikzcd}\]
    commutes, then $g_1 = g_2$.
    \end{definition}

    \begin{definition}
        \emph{(Alternatively, Riehl pg. 11)} An arrow $f : a \to b$ is \emph{monic} iff for any $C$-object $c$, post-composition with $f$ defines an injection $f_* : C(c,a) \to C(c,b)$.
        (Here $C(x,y)$ is the set of $C$-arrows from $x$ to $y$.)
    \end{definition}

    \subsection*{Exercises}
        For both exercises in this section, take the situation to be as follows:
        \[\begin{tikzcd}
            {s} && {a} && {b} && {c}
            \arrow["{f}", from=1-3, to=1-5]
            \arrow["{h_1}", from=1-1, to=1-3, curve={height=-6pt}]
            \arrow["{h_2}"', from=1-1, to=1-3, curve={height=6pt}]
            \arrow["{g}", from=1-5, to=1-7]
        \end{tikzcd}\]
        Where $f$ and $g$ are fixed, and $s, h_1, h_2$ are `any such' objects/arrows.
        \begin{enumerate}
        \item Suppose that $f$ and $g$ are both monic, and that 
        $g \circ (f \circ h_1) = g \circ (f \circ h_2)$.
        Since $g$ is monic, that implies $f \circ h_1 = f \circ h_2$.
        But since $f$ is monic, that implies $h_1 = h_2$. So using associativity and collapsing the chain of implication gives
        $$(g \circ f) \circ h_1 = (g \circ f) \circ h_2 \implies h_1 = h_2.$$
        Conclude $g \circ f$ is monic.

        \item Now suppose that $g \circ f$ is monic.
        If $f \circ h_1 = f \circ h_2$ then clearly $g \circ (f \circ h_1) = g \circ (f \circ h_2)$.
        Then $(g \circ f) \circ h_1 = (g \circ f) \circ h_2$ and since $g \circ f$ is monic, $h_1 = h_2$. So 
        $$f \circ h_1 = f \circ h_2 \implies h_1 = h_2,$$
        meaning $f$ is monic.
        \end{enumerate}

\section{Epic arrows}
    \begin{definition}
    If $f$ is \emph{epic} then commutativity of
        \[\begin{tikzcd}
            {a} && {b} && {c}
            \arrow["{f}", from=1-1, to=1-3]
            \arrow["{g_1}", from=1-3, to=1-5, curve={height=-6pt}]
            \arrow["{g_2}"', from=1-3, to=1-5, curve={height=6pt}]
        \end{tikzcd}\]
    implies $g_1 = g_2$.
    \end{definition}

    \begin{definition}
        \emph{(Alternatively, Riehl pg. 11)} An arrow $f : a \to b$ is \emph{epic} iff for any $C$-object $c$, pre-composition with $f$ defines an injection $f^* : C(b,x) \to C(a,c)$.
        (Here $C(x,y)$ is the set of $C$-arrows from $x$ to $y$.)
    \end{definition}

    Dually to the exercises proven in the previous section we have
    \begin{fact}
        If $f : a \to b$ and $g : b \to c$ are epic, then $g \o f : a \to c$ is epic.
    \end{fact}

    \begin{fact}
        If $g \o f : a \to c$ is epic, then $g : b \to c$ is epic.
    \end{fact}


\section{Iso arrows}
    \begin{definition}
        An arrow $f : a \to b$ is \emph{iso} if there exists another arrow $f\inv : b \to a$
        such that 
            $$f \circ f\inv = 1_b$$
            and
            $$f\inv \circ f = 1_a.$$
        This diagram commutes when the identity loops are included:
        \[\begin{tikzcd}
            {a} && {b}
            \arrow["{f}", from=1-1, to=1-3, curve={height=-12pt}]
            \arrow["{f\inv}", from=1-3, to=1-1, curve={height=-12pt}]
        \end{tikzcd}\]
    \end{definition}

    \begin{fact}
        If an arrow is iso then it is epic and monic, but the converse isn't necessarily true.
        The converse \emph{is} true in \textbf{Set} and any Topos.
    \end{fact}

    \subsection*{Exercises}
        \begin{enumerate}
            \item For any object $a$, the identity morphism $1_a$ is an inverse to itself and therefore is iso. Simply because
            $$1_a \o 1_a = 1_a.$$
            \[\begin{tikzcd}
                {a} && {a}
                \arrow["{1_a}", from=1-1, to=1-3, curve={height=-12pt}]
                \arrow["{1_a}", from=1-3, to=1-1, curve={height=-12pt}]
            \end{tikzcd}\]

            \item If $f : a \to b$ is iso then we can retrieve $f \inv$ and then plug it right into the definition and find
            $$f\inv \circ f = 1_a$$
            and
            $$f \circ f\inv = 1_b,$$
            indicating that $f\inv$ is iso.
            \[\begin{tikzcd}
                {a} && {b}
                \arrow["{f\inv}", from=1-1, to=1-3, curve={height=-12pt}]
                \arrow["{f}", from=1-3, to=1-1, curve={height=-12pt}]
            \end{tikzcd}\]

            \item With $f : a \to b$ and $g : b \to c$ both iso, the situation looks like the following:
            \[\begin{tikzcd}
                {a} && {b} && {c}
                \arrow["{f}", from=1-1, to=1-3, curve={height=-12pt}]
                \arrow["{f\inv}", from=1-3, to=1-1, curve={height=-12pt}]
                \arrow["{g\inv}", from=1-5, to=1-3, curve={height=-12pt}]
                \arrow["{g}", from=1-3, to=1-5, curve={height=-12pt}]
            \end{tikzcd}\]
            Now we find that 
            $$(f\inv \o g\inv) \o (g \o f)
            = f\inv \o (g\inv \o g) \o f
            = f\inv \o 1_b \o f
            = f\inv \o f
            = 1_a$$
            and
            $$(g \o f) \o (f\inv \o g\inv)
            = g  \o (f \o f\inv) \o g\inv
            = g \o 1_b \o g\inv
            = g \o g\inv
            = 1_c.$$
            Thus $(f\inv \o g\inv)$ acts as an inverse to $g \o f$,
            and $g \o f$ is iso.        
        \end{enumerate}

\section{Isomorphic objects}

    \begin{definition}
        Two $C$-objects $a$ and $b$ are \emph{isomorphic}, or 
        $$a \cong b$$ 
        if there exists an iso $C$-arrow
        $$f: a \to b.$$
    \end{definition}

    \begin{definition}
        A category $C$ is \emph{skeletal} if $a \cong b$ implies $a = b$.
    \end{definition}

    \subsection*{Exercises}
        \begin{enumerate}
            \item We wish to show that object isomorphism is an equivalence relation, or that it's reflexive, symmetric, and transitive. Fortunately the exercises from section 3 correspond exactly to these properties.
            \begin{enumerate}[(i)]
                \item $a \cong a$ since $1_a$ is iso.
                \item If $a \cong b$ then some $f : a \to b$ is iso, and therefore $f\inv : b \to a$ is iso and $b \cong a$.
                \item If $a \cong b$ and $b \cong c$ then we have iso arrows $f : a \to b$ and $g : b \to c$. Then $g \o f$ is iso, and $a \cong c$.
            \end{enumerate}

            \item Suppose $a$ and $b$ are two \textbf{Finord}-objects such that $a \cong b$. Then there is some $f : a \to b$ that is iso.
            Since \textbf{Finord} is a subcategory of \textbf{Set}, iso arrows correspond to bijective functions.
            Then $a$ and $b$ must have the same cardinality, but by the definition of \textbf{Finord} distinct objects have distinct cardinalities. So $a = b$ and \textbf{Finord} is skeletal.
            
        \end{enumerate}


\section{Initial objects}

    \begin{definition}
        An object $c$ is \emph{initial} if for every $C$-object $a$, there is exactly one arrow $f : c \to a$.

        \[\begin{tikzcd}
            &&& {\bullet} \\
            && {\bullet} & {\bullet} \\
            && {\bullet} & {\bullet} \\
            {c} && {\bullet} & {\bullet} \\
            && {\bullet} & {\bullet} \\
            && {\bullet} & {\bullet} \\
            &&& {\bullet}
            \arrow[from=4-1, to=2-3]
            \arrow[from=4-1, to=3-3]
            \arrow[from=4-1, to=4-3]
            \arrow[from=4-1, to=5-3]
            \arrow[from=4-1, to=6-3]
            \arrow[from=4-1, to=1-4, curve={height=-6pt}]
            \arrow[from=4-1, to=7-4, curve={height=6pt}]
            \arrow[from=4-1, to=6-4]
            \arrow[from=4-1, to=5-4]
            \arrow[from=4-1, to=4-4, curve={height=6pt}]
            \arrow[from=4-1, to=3-4]
            \arrow[from=4-1, to=2-4]
        \end{tikzcd}\]
    \end{definition}

\section{Terminal objects}
    \begin{definition}
        An object $c$ is \emph{terminal} if for every $C$-object $a$, there is exactly one arrow $f : a \to c$.       
        \[\begin{tikzcd}
            {\bullet} \\
            {\bullet} & {\bullet} \\
            {\bullet} & {\bullet} \\
            {\bullet} & {\bullet} && {c} \\
            {\bullet} & {\bullet} \\
            {\bullet} & {\bullet} \\
            {\bullet}
            \arrow[from=2-1, to=4-4]
            \arrow[from=3-1, to=4-4]
            \arrow[from=4-1, to=4-4, curve={height=6pt}]
            \arrow[from=5-1, to=4-4]
            \arrow[from=6-1, to=4-4]
            \arrow[from=1-1, to=4-4, curve={height=-6pt}]
            \arrow[from=7-1, to=4-4, curve={height=6pt}]
            \arrow[from=6-2, to=4-4]
            \arrow[from=5-2, to=4-4]
            \arrow[from=4-2, to=4-4]
            \arrow[from=3-2, to=4-4]
            \arrow[from=2-2, to=4-4]
        \end{tikzcd}\]
    \end{definition}

    \subsection*{Exercises}
        \begin{enumerate}
            \item Let $c_1$ and $c_2$ be terminal $C-objects$.
                \[\begin{tikzcd}
                    {\bullet} \\
                    {\bullet} & {\bullet} \\
                    {\bullet} & {\bullet} && {c_1} \\
                    {\bullet} & {\bullet} \\
                    {\bullet} & {\bullet} && {c_2} \\
                    {\bullet} & {\bullet} \\
                    {\bullet}
                    \arrow[from=2-1, to=3-4]
                    \arrow[from=3-1, to=3-4]
                    \arrow[from=4-1, to=3-4]
                    \arrow[from=5-1, to=3-4]
                    \arrow[from=6-1, to=3-4]
                    \arrow[from=1-1, to=3-4]
                    \arrow[from=7-1, to=3-4]
                    \arrow[from=6-2, to=3-4]
                    \arrow[from=5-2, to=3-4]
                    \arrow[from=4-2, to=3-4]
                    \arrow[from=3-2, to=3-4]
                    \arrow[from=2-2, to=3-4]
                    \arrow[from=1-1, to=5-4, curve={height=6pt}]
                    \arrow[from=2-2, to=5-4]
                    \arrow[from=2-1, to=5-4]
                    \arrow[from=3-1, to=5-4]
                    \arrow[from=3-2, to=5-4]
                    \arrow[from=4-1, to=5-4]
                    \arrow[from=4-2, to=5-4]
                    \arrow[from=5-1, to=5-4]
                    \arrow[from=5-2, to=5-4]
                    \arrow[from=6-1, to=5-4]
                    \arrow[from=6-2, to=5-4]
                    \arrow[from=7-1, to=5-4]
                    \arrow[from=3-4, to=5-4, shift left=1]
                    \arrow[from=5-4, to=3-4, shift left=1]
                \end{tikzcd}\]
            By terminality there is a unique arrow $f_1 : c_1 \to c_2$ and a unique arrow $f_2 : c_2 \to c_1$.
            Then by the category axiom, $f_2 \o f_1 : c_1 \to c_1$ and $f_1 \o f_2 : c_2 \to c_2$ must exist.
            But again by terminality, there is a unique arrow $1_{c_1} : c_1 \to c_1$ and $1_{c_2} : c_2 \to c_2$, so the composition of $f_1$ and $f_2$ must give the identity.
            Conclude $c_1 \cong c_2$.

            \item \begin{enumerate} [(i)]
                \item Terminal objects in \textbf{Set$^2$} are of the form $\langle \set{e_1}, \set{e_2}  \rangle$, or pairs of singleton sets.

                \item Terminal objects in \textbf{Set$^\rightarrow$}

                \item The terminal object in the poset \textbf{$(n, \leq)$} is the maximal element $n$, since $m \leq n$ for every $m$.
            \end{enumerate}

            \item Suppose $f : 1 \to a$ has its domain $1$ a terminal object, and $g_1, g_2$ are any two parallel arrows from $c \to 1$.
            \[\begin{tikzcd}
                {c} & {} & {1} && {a}
                \arrow["{f}", from=1-3, to=1-5]
                \arrow["{g_1}", from=1-1, to=1-3, curve={height=-6pt}]
                \arrow["{g_2}"', from=1-1, to=1-3, curve={height=6pt}]
            \end{tikzcd}\]
            Well, since $1$ is terminal the arrow from $c \to 1$ is unique and we see that $g_1 = g_2$,
            so regardless of whether $g_1 \o f = g_2 \o f$ holds (which it does), we can conclude $f$ is monic.
        \end{enumerate}

\section{Duality}
    Any category can be turned into its opposite category. So any statement about a category can be dualized with all the arrows reversed.

\section{Products}
    \begin{definition}
        Given $C$-objects $a$ and $b$, a \emph{product} is a $C$-object $a\times b$ and 2 $C$-arrows $pr_a, pr_b$.
        \[\begin{tikzcd}
            {a} & {} & {a \times b} && {b}
            \arrow["{pr_a}"', from=1-3, to=1-1]
            \arrow["{pr_b}", from=1-3, to=1-5]
        \end{tikzcd}\]

        For any $c, f, g$ configured as follows
        \[\begin{tikzcd}
            && {c} \\
            \\
            {a} & {} & {a \times b} && {b}
            \arrow["{pr_a}"', from=3-3, to=3-1]
            \arrow["{pr_b}", from=3-3, to=3-5]
            \arrow["{f}"', from=1-3, to=3-1]
            \arrow["{g}", from=1-3, to=3-5]
        \end{tikzcd}\]
        $f$ and $g$ determine a unique $h : c \to (a \times b)$ so that
        \[\begin{tikzcd}
            && {c} \\
            \\
            {a} & {} & {a \times b} && {b}
            \arrow["{pr_a}"', from=3-3, to=3-1]
            \arrow["{pr_b}", from=3-3, to=3-5]
            \arrow["{f}"', from=1-3, to=3-1]
            \arrow["{g}", from=1-3, to=3-5]
            \arrow["{h}", from=1-3, to=3-3, dashed]
        \end{tikzcd}\]
        commutes. This is denoted
        $$c := \langle f, g \rangle.$$
    \end{definition}

    \begin{fact}
        Any two products of $a$ and $b$, say $a\times_1 b$ and $a \times_2 b$, are isomorphic to each other. Consider that in the diagram
        \[\begin{tikzcd}
            && {a \times_2 b} \\
            {a} & {} &&& {b} \\
            && {a \times_1 b}
            \arrow["{pr_a}"', from=3-3, to=2-1]
            \arrow["{pr_b}", from=3-3, to=2-5]
            \arrow["{f}"', from=1-3, to=2-1]
            \arrow["{g}", from=1-3, to=2-5]
            \arrow["{h_1}"', from=1-3, to=3-3, curve={height=6pt}, dashed]
            \arrow["{h_2}"', from=3-3, to=1-3, curve={height=6pt}, dashed]
        \end{tikzcd}\]
        $h_1$ and $h_2$ are uniquely determined. But the arrow from a product to itself must also be unique, so the composition of $h_1$ and $h_2$ must give identities (depending on the order).
    \end{fact}

\end{document}  