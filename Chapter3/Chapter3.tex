\documentclass[12pt]{article}
\usepackage{../common}
\begin{document}

\chapter{3}{Arrows Instead of Epsilon}

\section{Monic arrows}
    Monic arrows are an abstraction of injective functions.
    \begin{definition}
    An arrow $f:a\to b$ in a category $C$ is \emph{monic} if for any $g_1, g_2$ with codomain $a$, the implication
    $$f \circ g_1 = f \circ g_2 \implies g_1 = g_2$$
    holds. Or, if the diagram
    \[\begin{tikzcd}
        {c} && {a} && {b}
        \arrow["{f}", from=1-3, to=1-5]
        \arrow["{g_1}", from=1-1, to=1-3, curve={height=-6pt}]
        \arrow["{g_2}", from=1-1, to=1-3, curve={height=6pt}]
    \end{tikzcd}\]
    commutes, then $g_1 = g_2$.
    \end{definition}

    \subsection*{Exercises}
        For both exercises in this section, take the situation to be as follows:
        \[\begin{tikzcd}
            {s} && {a} && {b} && {c}
            \arrow["{f}", from=1-3, to=1-5]
            \arrow["{h_1}", from=1-1, to=1-3, curve={height=-6pt}]
            \arrow["{h_2}", from=1-1, to=1-3, curve={height=6pt}]
            \arrow["{g}", from=1-5, to=1-7]
        \end{tikzcd}\]
        Where $f$ and $g$ are fixed, and $s, h_1, h_2$ are `any such' objects/arrows.
        \begin{enumerate}
        \item Suppose that $f$ and $g$ are both monic, and that 
        $g \circ (f \circ h_1) = g \circ (f \circ h_2)$.
        Since $g$ is monic, that implies $f \circ h_1 = f \circ h_2$.
        But since $f$ is monic, that implies $h_1 = h_2$. So using associativity and collapsing the chain of implication gives
        $$(g \circ f) \circ h_1 = (g \circ f) \circ h_2 \implies h_1 = h_2.$$
        Conclude $g \circ f$ is monic.

        \item Now suppose that $g \circ f$ is monic.
        If $f \circ h_1 = f \circ h_2$ then clearly $g \circ (f \circ h_1) = g \circ (f \circ h_2)$.
        Then $(g \circ f) \circ h_1 = (g \circ f) \circ h_2$ and since $g \circ f$ is monic, $h_1 = h_2$. So 
        $$f \circ h_1 = f \circ h_2 \implies h_1 = h_2,$$
        meaning $f$ is monic.
        \end{enumerate}

\section{Epic arrows}
    \begin{definition}
    If $f$ is \emph{epic} then commutativity of
        \[\begin{tikzcd}
            {a} && {b} && {c}
            \arrow["{f}", from=1-1, to=1-3]
            \arrow["{g_1}", from=1-3, to=1-5, curve={height=-6pt}]
            \arrow["{g_2}", from=1-3, to=1-5, curve={height=6pt}]
        \end{tikzcd}\]
    implies $g_1 = g_2$.
    \end{definition}

\section{Iso arrows}
    \begin{definition}
        An arrow $f : a \to b$ is \emph{iso} if there exists another arrow $f\inv : b \to a$
        such that 
            $$f \circ f\inv = 1_b$$
            and
            $$f\inv \circ f = 1_a.$$
        This diagram commutes when the identity loops are included:
        \[\begin{tikzcd}
            {a} && {b}
            \arrow["{f}", from=1-1, to=1-3, curve={height=-12pt}]
            \arrow["{f\inv}", from=1-3, to=1-1, curve={height=-12pt}]
        \end{tikzcd}\]
    \end{definition}

    \begin{fact}
        If an arrow is iso then it is epic and monic, but the converse isn't necessarily true.
        The converse \emph{is} true in \textbf{Set} and any Topos.
    \end{fact}

    \subsection{Exercises}
        \begin{enumerate}
            \item For any object $a$, the identity morphism $1_a$ is an inverse to itself and therefore is iso. Simply because
            $$1_a \o 1_a = 1_a.$$
            \[\begin{tikzcd}
                {a} && {a}
                \arrow["{1_a}", from=1-1, to=1-3, curve={height=-12pt}]
                \arrow["{1_a}", from=1-3, to=1-1, curve={height=-12pt}]
            \end{tikzcd}\]

            \item If $f : a \to b$ is iso then we can retrieve $f \inv$ and then plug it right into the definition and find
            $$f\inv \circ f = 1_a$$
            and
            $$f \circ f\inv = 1_b,$$
            indicating that $f\inv$ is iso.
            \[\begin{tikzcd}
                {a} && {b}
                \arrow["{f\inv}", from=1-1, to=1-3, curve={height=-12pt}]
                \arrow["{f}", from=1-3, to=1-1, curve={height=-12pt}]
            \end{tikzcd}\]

            \item With $f : a \to b$ and $g : b \to c$ both iso, the situation looks like the following:
            \[\begin{tikzcd}
                {a} && {b} && {c}
                \arrow["{f}", from=1-1, to=1-3, curve={height=-12pt}]
                \arrow["{f\inv}", from=1-3, to=1-1, curve={height=-12pt}]
                \arrow["{g\inv}", from=1-5, to=1-3, curve={height=-12pt}]
                \arrow["{g}", from=1-3, to=1-5, curve={height=-12pt}]
            \end{tikzcd}\]
            Now we find that 
            $$(f\inv \o g\inv) \o (g \o f)
            = f\inv \o (g\inv \o g) \o f
            = f\inv \o 1_b \o f
            = f\inv \o f
            = 1_a$$
            and
            $$(g \o f) \o (f\inv \o g\inv)
            = g  \o (f \o f\inv) \o g\inv
            = g \o 1_b \o g\inv
            = g \o g\inv
            = 1_c.$$
            Thus $(f\inv \o g\inv)$ acts as an inverse to $g \o f$,
            and $g \o f$ is iso.        
        \end{enumerate}

\section{Isomorphic objects}

    \begin{definition}
        Two $C$-objects $a$ and $b$ are \emph{isomorphic}, or 
        $$a \cong b$$ 
        if there exists an iso $C$-arrow
        $$f: a \to b.$$
    \end{definition}

    \begin{definition}
        A category $C$ is \emph{skeletal} if $a \cong b$ implies $a = b$.
    \end{definition}

    \subsection{Exercises}
        \begin{enumerate}
            \item We wish to show that object isomorphism is an equivalence relation, or that it's reflexive, symmetric, and transitive. Fortunately the exercises from section 3 correspond exactly to these properties.
            \begin{enumerate}[(i)]
                \item $a \cong a$ since $1_a$ is iso.
                \item If $a \cong b$ then some $f : a \to b$ is iso, and therefore $f\inv : b \to a$ is iso.
                \item If $a \cong b$ and $b \cong c$ then we have iso arrows $f : a \to b$ and $g : b \to c$. Then $g \o f$ is iso, and $a \cong c$.
            \end{enumerate}

            \item Suppose $a$ and $b$ are two \textbf{Finord}-objects such that $a \cong b$. Then there is some $f : a \to b$ that is iso.
            Since \textbf{Finord} is a subcategory of \textbf{Set}, iso arrows correspond to bijective functions.
            Then $a$ and $b$ must have the same cardinality, but by the definition of \textbf{Finord} distinct objects have distinct cardinalities. So $a = b$ and \textbf{Finord} is skeletal.
            
        \end{enumerate}



\end{document}