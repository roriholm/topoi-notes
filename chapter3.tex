\chapter{Arrows Instead of Epsilon}
\section{Monic arrows}
    Monic arrows are an abstraction of injective functions.
    \begin{defi}
    An arrow $f:a\to b$ in a category $C$ is \emph{monic} if for any $g_1, g_2$ with codomain $a$, the implication
    $$f \circ g_1 = f \circ g_2 \implies g_1 = g_2$$
    holds. Or, if the diagram
    \[\begin{tikzcd}
        {c} && {a} && {b}
        \arrow["{f}", from=1-3, to=1-5]
        \arrow["{g_1}", from=1-1, to=1-3, curve={height=-6pt}]
        \arrow["{g_2}"', from=1-1, to=1-3, curve={height=6pt}]
    \end{tikzcd}\]
    commutes, then $g_1 = g_2$.
    \end{defi}

    \begin{defi}
        \emph{(Alternatively, Riehl pg. 11)} An arrow $f : a \to b$ is \emph{monic} iff for any $C$-object $c$, post-composition with $f$ defines an injection $f_* : C(c,a) \to C(c,b)$.
        (Here $C(x,y)$ is the set of $C$-arrows from $x$ to $y$.)
    \end{defi}

    For both exercises in this section, take the situation to be as follows:
    \[\begin{tikzcd}
        {s} && {a} && {b} && {c}
        \arrow["{f}", from=1-3, to=1-5]
        \arrow["{h_1}", from=1-1, to=1-3, curve={height=-6pt}]
        \arrow["{h_2}"', from=1-1, to=1-3, curve={height=6pt}]
        \arrow["{g}", from=1-5, to=1-7]
    \end{tikzcd}\]
    Where $f$ and $g$ are fixed, and $s, h_1, h_2$ are `any such' objects/arrows.

    \begin{exercise}
        \item Suppose that $f$ and $g$ are both monic, and that 
        $g \circ (f \circ h_1) = g \circ (f \circ h_2)$.
        Since $g$ is monic, that implies $f \circ h_1 = f \circ h_2$.
        But since $f$ is monic, that implies $h_1 = h_2$. So using associativity and collapsing the chain of implication gives
        $$(g \circ f) \circ h_1 = (g \circ f) \circ h_2 \implies h_1 = h_2.$$
        Conclude $g \circ f$ is monic.
    \end{exercise}

    \begin{exercise}
        \item Now suppose that $g \circ f$ is monic.
        If $f \circ h_1 = f \circ h_2$ then clearly $g \circ (f \circ h_1) = g \circ (f \circ h_2)$.
        Then $(g \circ f) \circ h_1 = (g \circ f) \circ h_2$ and since $g \circ f$ is monic, $h_1 = h_2$. So 
        $$f \circ h_1 = f \circ h_2 \implies h_1 = h_2,$$
        meaning $f$ is monic.
    \end{exercise}

\section{Epic arrows}
    \begin{defi}
    If $f$ is \emph{epic} then commutativity of
        \[\begin{tikzcd}
            {a} && {b} && {c}
            \arrow["{f}", from=1-1, to=1-3]
            \arrow["{g_1}", from=1-3, to=1-5, curve={height=-6pt}]
            \arrow["{g_2}"', from=1-3, to=1-5, curve={height=6pt}]
        \end{tikzcd}\]
    implies $g_1 = g_2$.
    \end{defi}

    \begin{defi}
        \emph{(Alternatively, Riehl pg. 11)} An arrow $f : a \to b$ is \emph{epic} iff for any $C$-object $c$, pre-composition with $f$ defines an injection $f^* : C(b,x) \to C(a,c)$.
        (Here $C(x,y)$ is the set of $C$-arrows from $x$ to $y$.)
    \end{defi}

    Dually to the exercises proven in the previous section we have

    \begin{fact}
        If $f : a \to b$ and $g : b \to c$ are epic, then $g \o f : a \to c$ is epic.
    \end{fact}

    \begin{fact}
        If $g \o f : a \to c$ is epic, then $g : b \to c$ is epic.
    \end{fact}


\section{Iso arrows}
    \begin{defi}
        An arrow $f : a \to b$ is \emph{iso} if there exists another arrow $f\inv : b \to a$
        such that 
            $$f \circ f\inv = 1_b$$
            and
            $$f\inv \circ f = 1_a.$$
        This diagram commutes when the identity loops are included:
        \[\begin{tikzcd}
            {a} && {b}
            \arrow["{f}", from=1-1, to=1-3, curve={height=-12pt}]
            \arrow["{f\inv}", from=1-3, to=1-1, curve={height=-12pt}]
        \end{tikzcd}\]
    \end{defi}

    \begin{fact}
        If an arrow is iso then it is epic and monic, but the converse isn't necessarily true.
        The converse \emph{is} true in \textbf{Set} and any Topos.
    \end{fact}

    \begin{exercise} 
        For any object $a$, the identity morphism $1_a$ is an inverse to itself and therefore is iso. Simply because
        $$1_a \o 1_a = 1_a.$$
        \[\begin{tikzcd}
            {a} && {a}
            \arrow["{1_a}", from=1-1, to=1-3, curve={height=-12pt}]
            \arrow["{1_a}", from=1-3, to=1-1, curve={height=-12pt}]
        \end{tikzcd}\]
    \end{exercise}

    \begin{exercise}
        If $f : a \to b$ is iso then we can retrieve $f \inv$ and then plug it right into the definition and find
        $$f\inv \circ f = 1_a$$
        and
        $$f \circ f\inv = 1_b,$$
        indicating that $f\inv$ is iso.
        \[\begin{tikzcd}
            {a} && {b}
            \arrow["{f\inv}", from=1-1, to=1-3, curve={height=-12pt}]
            \arrow["{f}", from=1-3, to=1-1, curve={height=-12pt}]
        \end{tikzcd}\]
    \end{exercise}

    \begin{exercise}
        With $f : a \to b$ and $g : b \to c$ both iso, the situation looks like the following:
        \[\begin{tikzcd}
            {a} && {b} && {c}
            \arrow["{f}", from=1-1, to=1-3, curve={height=-12pt}]
            \arrow["{f\inv}", from=1-3, to=1-1, curve={height=-12pt}]
            \arrow["{g\inv}", from=1-5, to=1-3, curve={height=-12pt}]
            \arrow["{g}", from=1-3, to=1-5, curve={height=-12pt}]
        \end{tikzcd}\]
        Now we find that 
        $$(f\inv \o g\inv) \o (g \o f)
        = f\inv \o (g\inv \o g) \o f
        = f\inv \o 1_b \o f
        = f\inv \o f
        = 1_a$$
        and
        $$(g \o f) \o (f\inv \o g\inv)
        = g  \o (f \o f\inv) \o g\inv
        = g \o 1_b \o g\inv
        = g \o g\inv
        = 1_c.$$
        Thus $(f\inv \o g\inv)$ acts as an inverse to $g \o f$,
        and $g \o f$ is iso.        
    \end{exercise}


\section{Isomorphic objects}

    \begin{defi}
        Two $C$-objects $a$ and $b$ are \emph{isomorphic}, or 
        $$a \cong b$$ 
        if there exists an iso $C$-arrow
        $$f: a \to b.$$
    \end{defi}

    \begin{defi}
        A category $C$ is \emph{skeletal} if $a \cong b$ implies $a = b$.
    \end{defi}

    \begin{exercise}
        \item We wish to show that object isomorphism is an equivalence relation, or that it's reflexive, symmetric, and transitive. Fortunately the exercises from section 3 correspond exactly to these properties.
        \begin{enumerate}[(i)]
            \item $a \cong a$ since $1_a$ is iso.
            \item If $a \cong b$ then some $f : a \to b$ is iso, and therefore $f\inv : b \to a$ is iso and $b \cong a$.
            \item If $a \cong b$ and $b \cong c$ then we have iso arrows $f : a \to b$ and $g : b \to c$. Then $g \o f$ is iso, and $a \cong c$.
        \end{enumerate}
    \end{exercise}

    \begin{exercise}
        \item Suppose $a$ and $b$ are two \textbf{Finord}-objects such that $a \cong b$. Then there is some $f : a \to b$ that is iso.
        Since \textbf{Finord} is a subcategory of \textbf{Set}, iso arrows correspond to bijective functions.
        Then $a$ and $b$ must have the same cardinality, but by the definition of \textbf{Finord} distinct objects have distinct cardinalities. So $a = b$ and \textbf{Finord} is skeletal.
    \end{exercise}

\section{Initial objects}

    \begin{defi}
        An object $c$ is \emph{initial} if for every $C$-object $a$, there is exactly one arrow $f : c \to a$.

        % https://q.uiver.app/?q=WzAsMTMsWzAsMCwiYyJdLFs2LDEsIlxcYnVsbGV0Il0sWzUsMSwiXFxidWxsZXQiXSxbNywxLCJcXGJ1bGxldCJdLFs4LDEsIlxcYnVsbGV0Il0sWzgsMywiXFxidWxsZXQiXSxbOCwyLCJcXGJ1bGxldCJdLFs3LDIsIlxcYnVsbGV0Il0sWzYsMiwiXFxidWxsZXQiXSxbNSwyLCJcXGJ1bGxldCJdLFs1LDMsIlxcYnVsbGV0Il0sWzYsMywiXFxidWxsZXQiXSxbNywzLCJcXGJ1bGxldCJdLFswLDQsIiIsMCx7InN0eWxlIjp7ImJvZHkiOnsibmFtZSI6ImRhc2hlZCJ9fX1dLFswLDVdLFswLDJdLFswLDEsIiIsMix7InN0eWxlIjp7ImJvZHkiOnsibmFtZSI6ImRhc2hlZCJ9fX1dLFswLDMsIiIsMix7InN0eWxlIjp7ImJvZHkiOnsibmFtZSI6ImRhc2hlZCJ9fX1dLFswLDldLFswLDEwXSxbMCwxMV0sWzAsMTJdLFswLDhdLFswLDddLFswLDZdXQ==
        \[\begin{tikzcd}
            {c} \\
            &&&&& {\bullet} & {\bullet} & {\bullet} & {\bullet} \\
            &&&&& {\bullet} & {\bullet} & {\bullet} & {\bullet} \\
            &&&&& {\bullet} & {\bullet} & {\bullet} & {\bullet}
            \arrow[from=1-1, to=2-9, dashed]
            \arrow[from=1-1, to=4-9, dashed]
            \arrow[from=1-1, to=2-6, dashed]
            \arrow[from=1-1, to=2-7, dashed]
            \arrow[from=1-1, to=2-8, dashed]
            \arrow[from=1-1, to=3-6, dashed]
            \arrow[from=1-1, to=4-6, dashed]
            \arrow[from=1-1, to=4-7, dashed]
            \arrow[from=1-1, to=4-8, dashed]
            \arrow[from=1-1, to=3-7, dashed]
            \arrow[from=1-1, to=3-8, dashed]
            \arrow[from=1-1, to=3-9, dashed]
        \end{tikzcd}\]
    \end{defi}

\section{Terminal objects}
    \begin{defi}
        An object $c$ is \emph{terminal} if for every $C$-object $a$, there is exactly one arrow $f : a \to c$.       
        % https://q.uiver.app/?q=WzAsMTMsWzgsMywiYyJdLFsxLDAsIlxcYnVsbGV0Il0sWzAsMCwiXFxidWxsZXQiXSxbMiwwLCJcXGJ1bGxldCJdLFszLDAsIlxcYnVsbGV0Il0sWzMsMiwiXFxidWxsZXQiXSxbMywxLCJcXGJ1bGxldCJdLFsyLDEsIlxcYnVsbGV0Il0sWzEsMSwiXFxidWxsZXQiXSxbMCwxLCJcXGJ1bGxldCJdLFswLDIsIlxcYnVsbGV0Il0sWzEsMiwiXFxidWxsZXQiXSxbMiwyLCJcXGJ1bGxldCJdLFs0LDAsIiIsMix7InN0eWxlIjp7ImJvZHkiOnsibmFtZSI6ImRhc2hlZCJ9fX1dLFs1LDAsIiIsMCx7InN0eWxlIjp7ImJvZHkiOnsibmFtZSI6ImRhc2hlZCJ9fX1dLFsyLDAsIiIsMCx7InN0eWxlIjp7ImJvZHkiOnsibmFtZSI6ImRhc2hlZCJ9fX1dLFsxLDAsIiIsMCx7InN0eWxlIjp7ImJvZHkiOnsibmFtZSI6ImRhc2hlZCJ9fX1dLFszLDAsIiIsMCx7InN0eWxlIjp7ImJvZHkiOnsibmFtZSI6ImRhc2hlZCJ9fX1dLFs5LDAsIiIsMCx7InN0eWxlIjp7ImJvZHkiOnsibmFtZSI6ImRhc2hlZCJ9fX1dLFsxMCwwLCIiLDAseyJzdHlsZSI6eyJib2R5Ijp7Im5hbWUiOiJkYXNoZWQifX19XSxbMTEsMCwiIiwwLHsic3R5bGUiOnsiYm9keSI6eyJuYW1lIjoiZGFzaGVkIn19fV0sWzEyLDAsIiIsMCx7InN0eWxlIjp7ImJvZHkiOnsibmFtZSI6ImRhc2hlZCJ9fX1dLFs4LDAsIiIsMCx7InN0eWxlIjp7ImJvZHkiOnsibmFtZSI6ImRhc2hlZCJ9fX1dLFs3LDAsIiIsMCx7InN0eWxlIjp7ImJvZHkiOnsibmFtZSI6ImRhc2hlZCJ9fX1dLFs2LDAsIiIsMCx7InN0eWxlIjp7ImJvZHkiOnsibmFtZSI6ImRhc2hlZCJ9fX1dXQ==
        \[\begin{tikzcd}
            {\bullet} & {\bullet} & {\bullet} & {\bullet} \\
            {\bullet} & {\bullet} & {\bullet} & {\bullet} \\
            {\bullet} & {\bullet} & {\bullet} & {\bullet} \\
            &&&&&&&& {c}
            \arrow[from=1-4, to=4-9, dashed]
            \arrow[from=3-4, to=4-9, dashed]
            \arrow[from=1-1, to=4-9, dashed]
            \arrow[from=1-2, to=4-9, dashed]
            \arrow[from=1-3, to=4-9, dashed]
            \arrow[from=2-1, to=4-9, dashed]
            \arrow[from=3-1, to=4-9, dashed]
            \arrow[from=3-2, to=4-9, dashed]
            \arrow[from=3-3, to=4-9, dashed]
            \arrow[from=2-2, to=4-9, dashed]
            \arrow[from=2-3, to=4-9, dashed]
            \arrow[from=2-4, to=4-9, dashed]
        \end{tikzcd}\]
    \end{defi}

    \begin{exercise} \label{initial objects unique up to isomorphism}
        Let $c_1$ and $c_2$ be terminal $C-objects$.
            % https://q.uiver.app/?q=WzAsMTQsWzgsNCwiY18yIl0sWzEsMSwiXFxidWxsZXQiXSxbMCwxLCJcXGJ1bGxldCJdLFsyLDEsIlxcYnVsbGV0Il0sWzMsMSwiXFxidWxsZXQiXSxbMywzLCJcXGJ1bGxldCJdLFszLDIsIlxcYnVsbGV0Il0sWzIsMiwiXFxidWxsZXQiXSxbMSwyLCJcXGJ1bGxldCJdLFswLDIsIlxcYnVsbGV0Il0sWzAsMywiXFxidWxsZXQiXSxbMSwzLCJcXGJ1bGxldCJdLFsyLDMsIlxcYnVsbGV0Il0sWzgsMCwiY18xIl0sWzQsMCwiIiwyLHsic3R5bGUiOnsiYm9keSI6eyJuYW1lIjoiZGFzaGVkIn19fV0sWzUsMCwiIiwwLHsic3R5bGUiOnsiYm9keSI6eyJuYW1lIjoiZGFzaGVkIn19fV0sWzIsMCwiIiwwLHsic3R5bGUiOnsiYm9keSI6eyJuYW1lIjoiZGFzaGVkIn19fV0sWzEsMCwiIiwwLHsic3R5bGUiOnsiYm9keSI6eyJuYW1lIjoiZGFzaGVkIn19fV0sWzMsMCwiIiwwLHsic3R5bGUiOnsiYm9keSI6eyJuYW1lIjoiZGFzaGVkIn19fV0sWzksMCwiIiwwLHsic3R5bGUiOnsiYm9keSI6eyJuYW1lIjoiZGFzaGVkIn19fV0sWzEwLDAsIiIsMCx7InN0eWxlIjp7ImJvZHkiOnsibmFtZSI6ImRhc2hlZCJ9fX1dLFsxMSwwLCIiLDAseyJzdHlsZSI6eyJib2R5Ijp7Im5hbWUiOiJkYXNoZWQifX19XSxbMTIsMCwiIiwwLHsic3R5bGUiOnsiYm9keSI6eyJuYW1lIjoiZGFzaGVkIn19fV0sWzgsMCwiIiwwLHsic3R5bGUiOnsiYm9keSI6eyJuYW1lIjoiZGFzaGVkIn19fV0sWzcsMCwiIiwwLHsic3R5bGUiOnsiYm9keSI6eyJuYW1lIjoiZGFzaGVkIn19fV0sWzYsMCwiIiwwLHsic3R5bGUiOnsiYm9keSI6eyJuYW1lIjoiZGFzaGVkIn19fV0sWzIsMTMsIiIsMix7InN0eWxlIjp7ImJvZHkiOnsibmFtZSI6ImRhc2hlZCJ9fX1dLFs0LDEzLCIiLDAseyJzdHlsZSI6eyJib2R5Ijp7Im5hbWUiOiJkYXNoZWQifX19XSxbMywxMywiIiwwLHsic3R5bGUiOnsiYm9keSI6eyJuYW1lIjoiZGFzaGVkIn19fV0sWzEsMTMsIiIsMCx7InN0eWxlIjp7ImJvZHkiOnsibmFtZSI6ImRhc2hlZCJ9fX1dLFs2LDEzLCIiLDAseyJzdHlsZSI6eyJib2R5Ijp7Im5hbWUiOiJkYXNoZWQifX19XSxbNywxMywiIiwwLHsic3R5bGUiOnsiYm9keSI6eyJuYW1lIjoiZGFzaGVkIn19fV0sWzksMTMsIiIsMCx7InN0eWxlIjp7ImJvZHkiOnsibmFtZSI6ImRhc2hlZCJ9fX1dLFs1LDEzLCIiLDAseyJzdHlsZSI6eyJib2R5Ijp7Im5hbWUiOiJkYXNoZWQifX19XSxbMTIsMTMsIiIsMCx7InN0eWxlIjp7ImJvZHkiOnsibmFtZSI6ImRhc2hlZCJ9fX1dLFsxMSwxMywiIiwwLHsic3R5bGUiOnsiYm9keSI6eyJuYW1lIjoiZGFzaGVkIn19fV0sWzEwLDEzLCIiLDAseyJzdHlsZSI6eyJib2R5Ijp7Im5hbWUiOiJkYXNoZWQifX19XSxbMTMsMCwiIiwwLHsib2Zmc2V0IjotMSwic3R5bGUiOnsiYm9keSI6eyJuYW1lIjoiZGFzaGVkIn19fV0sWzAsMTMsIiIsMCx7Im9mZnNldCI6LTEsInN0eWxlIjp7ImJvZHkiOnsibmFtZSI6ImRhc2hlZCJ9fX1dXQ==
            \[\begin{tikzcd}
                &&&&&&&& {c_1} \\
                {\bullet} & {\bullet} & {\bullet} & {\bullet} \\
                {\bullet} & {\bullet} & {\bullet} & {\bullet} \\
                {\bullet} & {\bullet} & {\bullet} & {\bullet} \\
                &&&&&&&& {c_2}
                \arrow[from=2-4, to=5-9, dashed]
                \arrow[from=4-4, to=5-9, dashed]
                \arrow[from=2-1, to=5-9, dashed]
                \arrow[from=2-2, to=5-9, dashed]
                \arrow[from=2-3, to=5-9, dashed]
                \arrow[from=3-1, to=5-9, dashed]
                \arrow[from=4-1, to=5-9, dashed]
                \arrow[from=4-2, to=5-9, dashed]
                \arrow[from=4-3, to=5-9, dashed]
                \arrow[from=3-2, to=5-9, dashed]
                \arrow[from=3-3, to=5-9, dashed]
                \arrow[from=3-4, to=5-9, dashed]
                \arrow[from=2-1, to=1-9, dashed]
                \arrow[from=2-4, to=1-9, dashed]
                \arrow[from=2-3, to=1-9, dashed]
                \arrow[from=2-2, to=1-9, dashed]
                \arrow[from=3-4, to=1-9, dashed]
                \arrow[from=3-3, to=1-9, dashed]
                \arrow[from=3-1, to=1-9, dashed]
                \arrow[from=4-4, to=1-9, dashed]
                \arrow[from=4-3, to=1-9, dashed]
                \arrow[from=4-2, to=1-9, dashed]
                \arrow[from=4-1, to=1-9, dashed]
                \arrow[from=1-9, to=5-9, shift left=1, dashed]
                \arrow[from=5-9, to=1-9, shift left=1, dashed]
            \end{tikzcd}\]
        By terminality there is a unique arrow $f_1 : c_1 \to c_2$ and a unique arrow $f_2 : c_2 \to c_1$.
        Then by the category axiom, $f_2 \o f_1 : c_1 \to c_1$ and $f_1 \o f_2 : c_2 \to c_2$ must exist.
        But again by terminality, there is a unique arrow $1_{c_1} : c_1 \to c_1$ and $1_{c_2} : c_2 \to c_2$, so the composition of $f_1$ and $f_2$ must give the identity.
        Conclude $c_1 \cong c_2$.
    \end{exercise}

    \begin{exercise}
        \begin{enumerate} [(i)]
            \item Terminal objects in \textbf{Set$^2$} are of the form $\langle \set{e_1}, \set{e_2}  \rangle$, or pairs of singleton sets.

            \item Terminal objects in \textbf{Set$^\rightarrow$} are arrows with singleton sets as domain and codomain.

            \item The terminal object in the poset \textbf{$(n, \leq)$} is the maximal element $n$, since $m \leq n$ for every $m$.
        \end{enumerate}
    \end{exercise}

    \begin{exercise} Suppose $f : 1 \to a$ has its domain $1$ a terminal object, and $g_1, g_2$ are any two parallel arrows from $c \to 1$.
        \[\begin{tikzcd}
            {c} & {} & {1} && {a}
            \arrow["{f}", from=1-3, to=1-5]
            \arrow["{g_1}", from=1-1, to=1-3, curve={height=-6pt}]
            \arrow["{g_2}"', from=1-1, to=1-3, curve={height=6pt}]
        \end{tikzcd}\]
        Well, since $1$ is terminal the arrow from $c \to 1$ is unique and we see that $g_1 = g_2$,
        so regardless of whether $g_1 \o f = g_2 \o f$ holds (which it does), we can conclude $f$ is monic.
    \end{exercise}

\section{Duality}
    Any category can be turned into its opposite category. So any statement about a category can be dualized with all the arrows reversed.

\section{Products}
    \begin{defi} \label{product}
        Given $C$-objects $a$ and $b$, a \emph{product} is a $C$-object $a\times b$ and 2 $C$-arrows $pr_a, pr_b$.
        \[\begin{tikzcd}
            {a} & {} & {a \times b} && {b}
            \arrow["{pr_a}"', from=1-3, to=1-1]
            \arrow["{pr_b}", from=1-3, to=1-5]
        \end{tikzcd}\]

        For any $c, f, g$ configured as follows
        \[\begin{tikzcd}
            && {c} \\
            \\
            {a} & {} & {a \times b} && {b}
            \arrow["{pr_a}"', from=3-3, to=3-1]
            \arrow["{pr_b}", from=3-3, to=3-5]
            \arrow["{f}"', from=1-3, to=3-1]
            \arrow["{g}", from=1-3, to=3-5]
        \end{tikzcd}\]
        $f$ and $g$ determine a unique $h : c \to (a \times b)$ so that
        \[\begin{tikzcd}
            && {c} \\
            \\
            {a} & {} & {a \times b} && {b}
            \arrow["{pr_a}"', from=3-3, to=3-1]
            \arrow["{pr_b}", from=3-3, to=3-5]
            \arrow["{f}"', from=1-3, to=3-1]
            \arrow["{g}", from=1-3, to=3-5]
            \arrow["{h}", from=1-3, to=3-3, dashed]
        \end{tikzcd}\]
        commutes. This is denoted
        $$h := \langle f, g \rangle.$$
    \end{defi}

    \begin{fact} \label{product has one identity}
        If $c$ is a product $a \times b$, any arrow $f : c \to c$, $f$ must be the identity $1_c$.
        First observe that the identity must exist. Then plug $c$ into \cref{product} to see that $f$ must be the unique arrow with that domain and codomain.
        % https://q.uiver.app/?q=WzAsNSxbNCwxLCJiIl0sWzEsMV0sWzAsMSwiYSJdLFsyLDEsImEgXFx0aW1lcyBiIl0sWzIsMCwiYSBcXHRpbWVzIGIiXSxbMywyLCJwcl9hIiwyXSxbMywwLCJwcl9iIl0sWzQsMiwiZiIsMl0sWzQsMCwiZyJdLFszLDQsIjFfe2EgXFx0aW1lcyBifSIsMix7InN0eWxlIjp7InRhaWwiOnsibmFtZSI6ImFycm93aGVhZCJ9LCJib2R5Ijp7Im5hbWUiOiJkYXNoZWQifX19XV0=
        \[\begin{tikzcd}
            && {c} \\
            {a} & {} & {c} && {b}
            \arrow["{pr_a}"', from=2-3, to=2-1]
            \arrow["{pr_b}", from=2-3, to=2-5]
            \arrow["{f}"', from=1-3, to=2-1]
            \arrow["{g}", from=1-3, to=2-5]
            \arrow["{1_c}"', from=2-3, to=1-3, dashed, tail reversed]
        \end{tikzcd}\]
    \end{fact}

    \begin{fact} \label{products are isomorphic}
        Any two products of $a$ and $b$, say $a\times_1 b$ and $a \times_2 b$, are isomorphic to each other. Consider that in the diagram
        % https://q.uiver.app/?q=WzAsNSxbNCwxLCJiIl0sWzEsMV0sWzAsMSwiYSJdLFsyLDIsImEgXFx0aW1lc18xIGIiXSxbMiwwLCJhIFxcdGltZXNfMiBiIl0sWzMsMiwicHJfYSIsMl0sWzMsMCwicHJfYiJdLFs0LDIsImYiLDJdLFs0LDAsImciXSxbNCwzLCJoXzEiLDIseyJjdXJ2ZSI6MSwic3R5bGUiOnsiYm9keSI6eyJuYW1lIjoiZGFzaGVkIn19fV0sWzMsNCwiaF8yIiwyLHsiY3VydmUiOjEsInN0eWxlIjp7ImJvZHkiOnsibmFtZSI6ImRhc2hlZCJ9fX1dXQ==
        \[\begin{tikzcd}
            && {a \times_2 b} \\
            {a} & {} &&& {b} \\
            && {a \times_1 b}
            \arrow["{pr_a}"', from=3-3, to=2-1]
            \arrow["{pr_b}", from=3-3, to=2-5]
            \arrow["{f}"', from=1-3, to=2-1]
            \arrow["{g}", from=1-3, to=2-5]
            \arrow["{h_1}"', from=1-3, to=3-3, curve={height=6pt}, dashed]
            \arrow["{h_2}"', from=3-3, to=1-3, curve={height=6pt}, dashed]
        \end{tikzcd}\]
        $h_1$ and $h_2$ are uniquely determined by symmetric applications of \cref{product}. But by \cref{product has one identity}, composition of $h_1$ and $h_2$ must give identities.
    \end{fact}

    \begin{exercise} \label{projections combine to identity}
        The fact that $\langle pr_a, pr_b \rangle = 1_{a\times b}$ follows as a special case of \cref{product has one identity}, by plugging in the projection functions.
        % https://q.uiver.app/?q=WzAsNSxbNCwxLCJiIl0sWzEsMV0sWzAsMSwiYSJdLFsyLDEsImEgXFx0aW1lcyBiIl0sWzIsMCwiYSBcXHRpbWVzIGIiXSxbMywyLCJwcl9hIiwyXSxbMywwLCJwcl9iIl0sWzQsMiwicHJfYSIsMl0sWzQsMCwicHJfYiJdLFszLDQsIjFfe2EgXFx0aW1lcyBifSIsMix7InN0eWxlIjp7InRhaWwiOnsibmFtZSI6ImFycm93aGVhZCJ9LCJib2R5Ijp7Im5hbWUiOiJkYXNoZWQifX19XV0=
        \[\begin{tikzcd}
            && {a \times b} \\
            {a} & {} & {a \times b} && {b}
            \arrow["{pr_a}"', from=2-3, to=2-1]
            \arrow["{pr_b}", from=2-3, to=2-5]
            \arrow["{pr_a}"', from=1-3, to=2-1]
            \arrow["{pr_b}", from=1-3, to=2-5]
            \arrow["{1_{a \times b}}"', from=2-3, to=1-3, dashed, tail reversed]
        \end{tikzcd}\]
    \end{exercise}

    \begin{exercise}
        Suppose we have parallel $f,k : c \too a$ and $g, h : c \too b$ and $p : c \to a \times b$ such that 
        $p = \langle f, g \rangle = \langle k, h \rangle$.
        Then $f = pr_a \circ p$ and $k = pr_a \circ p$. It doesn't take any special cancellation rules to see that identically $f = k$. Similarly $g=h$.
        % https://q.uiver.app/?q=WzAsNSxbNCwyLCJiIl0sWzEsMl0sWzAsMiwiYSJdLFsyLDIsImEgXFx0aW1lcyBiIl0sWzIsMCwiYyJdLFszLDIsInByX2EiLDJdLFszLDAsInByX2IiXSxbNCwyLCJmIiwyLHsiY3VydmUiOjJ9XSxbNCwwLCJnIiwwLHsiY3VydmUiOi0yfV0sWzQsMywicCIsMCx7InN0eWxlIjp7ImJvZHkiOnsibmFtZSI6ImRhc2hlZCJ9fX1dLFs0LDIsImsiXSxbNCwwLCJoIiwyXV0=
        \[\begin{tikzcd}
            && {c} \\
            \\
            {a} & {} & {a \times b} && {b}
            \arrow["{pr_a}"', from=3-3, to=3-1]
            \arrow["{pr_b}", from=3-3, to=3-5]
            \arrow["{f}"', from=1-3, to=3-1, curve={height=12pt}]
            \arrow["{g}", from=1-3, to=3-5, curve={height=-12pt}]
            \arrow["{p}", from=1-3, to=3-3, dashed]
            \arrow["{k}", from=1-3, to=3-1]
            \arrow["{h}"', from=1-3, to=3-5]
        \end{tikzcd}\]
    \end{exercise}

    \begin{exercise}
        Suppose the situation is as follows.
        % https://q.uiver.app/?q=WzAsNixbNCw0LCJiIl0sWzEsNF0sWzAsNCwiYSJdLFsyLDQsImEgXFx0aW1lcyBiIl0sWzIsMiwiYyJdLFsyLDAsImQiXSxbMywyLCJwcl9hIiwyXSxbMywwLCJwcl9iIl0sWzQsMiwiZiIsMix7ImN1cnZlIjoyfV0sWzQsMCwiZyIsMCx7ImN1cnZlIjotMn1dLFs0LDMsIlxcbGFuZ2xlIGYsIGcgXFxyYW5nbGUiLDAseyJzdHlsZSI6eyJib2R5Ijp7Im5hbWUiOiJkYXNoZWQifX19XSxbNSw0LCJoIiwyXV0=
        \[\begin{tikzcd}
            && {d} \\
            \\
            && {c} \\
            \\
            {a} & {} & {a \times b} && {b}
            \arrow["{pr_a}"', from=5-3, to=5-1]
            \arrow["{pr_b}", from=5-3, to=5-5]
            \arrow["{f}"', from=3-3, to=5-1, curve={height=12pt}]
            \arrow["{g}", from=3-3, to=5-5, curve={height=-12pt}]
            \arrow["{\langle f, g \rangle}", from=3-3, to=5-3, dashed]
            \arrow["{h}"', from=1-3, to=3-3]
        \end{tikzcd}\]
        Then by compositionality there must exist $h \circ f : d \to a$ and $g \circ f : d \to b$.
        There also must exist $h \circ \langle f, g \rangle : d \to a \times b$.
        By collapsing the diagram to 
        % https://q.uiver.app/?q=WzAsNSxbNCwyLCJiIl0sWzEsMl0sWzAsMiwiYSJdLFsyLDIsImEgXFx0aW1lcyBiIl0sWzIsMCwiZCJdLFszLDIsInByX2EiLDJdLFszLDAsInByX2IiXSxbNCwyLCJmIFxcY2lyYyBoIiwyLHsiY3VydmUiOjJ9XSxbNCwwLCJnIFxcY2lyYyBoIiwwLHsiY3VydmUiOi0yfV0sWzQsMywiXFxsYW5nbGUgZiBcXGNpcmMgaCwgZyBcXGNpcmMgaCBcXHJhbmdsZSIsMCx7InN0eWxlIjp7ImJvZHkiOnsibmFtZSI6ImRhc2hlZCJ9fX1dXQ==
        \[\begin{tikzcd}
            && {d} \\
            \\
            {a} & {} & {a \times b} && {b}
            \arrow["{pr_a}"', from=3-3, to=3-1]
            \arrow["{pr_b}", from=3-3, to=3-5]
            \arrow["{f \circ h}"', from=1-3, to=3-1, curve={height=12pt}]
            \arrow["{g \circ h}", from=1-3, to=3-5, curve={height=-12pt}]
            \arrow["{\langle f \circ h, g \circ h \rangle}", from=1-3, to=3-3, dashed]
        \end{tikzcd}\]
        we see the arrow from $ d \to a \times b$ must be unique, and therefore
        $$\langle f\circ h, g\circ h \rangle = h \circ \langle f, g \rangle.$$
    \end{exercise}

    \begin{exercise} \label{a = a * 1}
        Suppose a category $C$ has a terminal object $t$, and products. Let $a$ be a $C$-object and consider the product $a \times t$.
        % https://q.uiver.app/?q=WzAsNCxbNCwwLCIxIl0sWzEsMF0sWzAsMCwiYSJdLFsyLDAsImEgXFx0aW1lcyAxIl0sWzMsMl0sWzMsMF1d
        \[\begin{tikzcd}
            {a} & {} & {a \times t} && {t}
            \arrow[from=1-3, to=1-1]
            \arrow[from=1-3, to=1-5]
        \end{tikzcd}\]
        Plugging $a$ into the product definition using the arrows given by $1_a$ and $!$ (the unique arrow to $t$)
        yields the unique arrow $h = \langle 1_a, ! \rangle$.
        % https://q.uiver.app/?q=WzAsNSxbNCwyLCJ0Il0sWzEsMl0sWzAsMiwiYSJdLFsyLDIsImEgXFx0aW1lcyB0Il0sWzIsMCwiYSJdLFszLDIsInByX2EiXSxbMywwXSxbNCwyLCIxX2EiLDJdLFs0LDAsIiEiXSxbNCwzLCJoIiwwLHsic3R5bGUiOnsiYm9keSI6eyJuYW1lIjoiZGFzaGVkIn19fV1d
        \[\begin{tikzcd}
            && {a} \\
            \\
            {a} & {} & {a \times t} && {t}
            \arrow["{pr_a}", from=3-3, to=3-1]
            \arrow[from=3-3, to=3-5]
            \arrow["{1_a}"', from=1-3, to=3-1]
            \arrow["{!}", from=1-3, to=3-5]
            \arrow["{h}", from=1-3, to=3-3, dashed]
        \end{tikzcd}\]
        By definition we have that $pr_a \o h = 1_a$. And since $h \o pr_a$ maps $a \times t \to a \times t$ it follows from \cref{product has one identity} that $h \o pr_a = 1_{a \times t}$. Given these two iso arrows we conclude
        $$a \cong a \times t.$$
    \end{exercise}

    \begin{example}
        As a specific example of \cref{a = a * 1}, let's work in \textbf{Set} where $A$ is any set, and our terminal set $t$ is any singleton set $\set{b}$.
        % https://q.uiver.app/?q=WzAsNSxbNCwyLCJcXHsgYiBcXH0iXSxbMSwyXSxbMCwyLCJBIl0sWzIsMiwiQSBcXHRpbWVzIFxceyBiIFxcfSJdLFsyLDAsIkEiXSxbMywyLCJwcl9hIl0sWzMsMF0sWzQsMiwiMV9hIiwyXSxbNCwwLCIhIl0sWzQsMywiaCIsMCx7InN0eWxlIjp7ImJvZHkiOnsibmFtZSI6ImRhc2hlZCJ9fX1dXQ==
        \[\begin{tikzcd}
            && {A} \\
            \\
            {A} & {} & {A \times \{ b \}} && {\{ b \}}
            \arrow["{pr_a}", from=3-3, to=3-1]
            \arrow["{pr_b}"', from=3-3, to=3-5]
            \arrow["{1_a}"', from=1-3, to=3-1]
            \arrow["{!}", from=1-3, to=3-5]
            \arrow["{h}", from=1-3, to=3-3, dashed]
        \end{tikzcd}\]
        Now we can explicitly say what all of our functions do:
        $$1_a(x) = x$$
        $$!(x) = b$$
        $$pr_a\left( \langle a, b \rangle \right) = a$$
        $$pr_b\left( \langle a, b \rangle \right) = b$$
        $$h(x) = \langle x, b \rangle$$
        The fact that $pr_a(h(x)) = x$ and $h(pr_a(\langle a, b \rangle)) = \langle a, b \rangle$ gives the isomorphism
        $$A \cong A \times \set{b}.$$
        Intuitively, elements in $A$ can be placed in one-to-one correspondence with elements in $A \times \set{b}$
        by simply sending $x$ to the tuple $\langle x, b \rangle$.

    \end{example}

    \begin{defi} \label{product arrow}
        Given two products $a \times b$ and $c \times d$, and arrows $f : a \to c$ and $g : b \to d$
        % https://q.uiver.app/?q=WzAsOCxbMiw2XSxbNCwwLCJiIl0sWzQsMiwiZCJdLFswLDAsImEiXSxbMCwyLCJjIl0sWzIsMCwiYSBcXHRpbWVzIGIiXSxbMiwyLCJjIFxcdGltZXMgZCJdLFsyLDFdLFszLDQsImYiXSxbMSwyLCJnIl0sWzUsMywicHJfYSJdLFs1LDEsInByX2IiLDJdLFs2LDQsInByX2MiLDJdLFs2LDIsInByX2QiXV0=
        \[\begin{tikzcd}
            {a} && {a \times b} && {b} \\
            && {} \\
            {c} && {c \times d} && {d} \\
            \\
            \\
            \\
            && {}
            \arrow["{f}", from=1-1, to=3-1]
            \arrow["{g}", from=1-5, to=3-5]
            \arrow["{pr_a}", from=1-3, to=1-1]
            \arrow["{pr_b}"', from=1-3, to=1-5]
            \arrow["{pr_c}"', from=3-3, to=3-1]
            \arrow["{pr_d}", from=3-3, to=3-5]
        \end{tikzcd}\]
        the unique \emph{product arrow} $(f \times g) : a \times b \to c \times d$ is found as $\langle f \circ pr_a, g \circ pr_b \rangle$.
        % https://q.uiver.app/?q=WzAsOCxbMiw2XSxbNCwwLCJiIl0sWzQsMiwiZCJdLFswLDAsImEiXSxbMCwyLCJjIl0sWzIsMCwiYSBcXHRpbWVzIGIiXSxbMiwyLCJjIFxcdGltZXMgZCJdLFsyLDFdLFszLDQsImYiXSxbMSwyLCJnIl0sWzUsMywicHJfYSJdLFs1LDEsInByX2IiLDJdLFs2LDQsInByX2MiLDJdLFs2LDIsInByX2QiXSxbNSw2LCJmXFx0aW1lcyBnIiwxLHsic3R5bGUiOnsiYm9keSI6eyJuYW1lIjoiZGFzaGVkIn19fV1d
        \[\begin{tikzcd}
            {a} && {a \times b} && {b} \\
            && {} \\
            {c} && {c \times d} && {d} \\
            \\
            \\
            \\
            && {}
            \arrow["{f}", from=1-1, to=3-1]
            \arrow["{g}", from=1-5, to=3-5]
            \arrow["{pr_a}", from=1-3, to=1-1]
            \arrow["{pr_b}"', from=1-3, to=1-5]
            \arrow["{pr_c}"', from=3-3, to=3-1]
            \arrow["{pr_d}", from=3-3, to=3-5]
            \arrow["{f\times g}" description, from=1-3, to=3-3, dashed]
        \end{tikzcd}\]
    \end{defi}

    \begin{exercise}
        In the reflexive case we consider the product arrow $1_a \times 1_b$.
        % https://q.uiver.app/?q=WzAsOCxbMiw2XSxbNCwwLCJiIl0sWzQsMiwiYiJdLFswLDAsImEiXSxbMCwyLCJhIl0sWzIsMCwiYSBcXHRpbWVzIGIiXSxbMiwyLCJhIFxcdGltZXMgYiJdLFsyLDFdLFszLDQsIjFfYSJdLFsxLDIsIjFfYiJdLFs1LDMsInByX2EiXSxbNSwxLCJwcl9iIiwyXSxbNiw0LCJwcl9hIiwyXSxbNiwyLCJwcl9iIl0sWzUsNiwiMV9hIFxcdGltZXMgMV9iIiwxLHsic3R5bGUiOnsiYm9keSI6eyJuYW1lIjoiZGFzaGVkIn19fV1d
        \[\begin{tikzcd}
            {a} && {a \times b} && {b} \\
            && {} \\
            {a} && {a \times b} && {b} \\
            \\
            \\
            \\
            && {}
            \arrow["{1_a}", from=1-1, to=3-1]
            \arrow["{1_b}", from=1-5, to=3-5]
            \arrow["{pr_a}", from=1-3, to=1-1]
            \arrow["{pr_b}"', from=1-3, to=1-5]
            \arrow["{pr_a}"', from=3-3, to=3-1]
            \arrow["{pr_b}", from=3-3, to=3-5]
            \arrow["{1_a \times 1_b}" description, from=1-3, to=3-3, dashed]
        \end{tikzcd}\]
        By definition we have
        $$1_a \times 1_b = \langle 1_a \o pr_a, 1_b \o pr_b \rangle
        = \langle pr_a, pr_b \rangle.$$
        Then applying \cref{projections combine to identity} from here gives the desired result
        $$1_a \times 1_b = 1_{a \times b}.$$

    \end{exercise}

    \begin{exercise}
        The isomorphism $a \times b \cong b \times a$ follows by plugging each object into \cref{product} in relation to the other.
        In this way the two unique (dashed) arrows are found:
        % https://q.uiver.app/?q=WzAsOCxbMiw2XSxbNCwwLCJiIl0sWzQsMiwiYSJdLFswLDAsImEiXSxbMCwyLCJiIl0sWzIsMCwiYSBcXHRpbWVzIGIiXSxbMiwyLCJiIFxcdGltZXMgYSJdLFsyLDFdLFs1LDNdLFs1LDFdLFs2LDRdLFs2LDJdLFszLDJdLFsxLDRdLFs1LDYsIiIsMSx7ImN1cnZlIjoxLCJzdHlsZSI6eyJib2R5Ijp7Im5hbWUiOiJkYXNoZWQifX19XSxbNiw1LCIiLDEseyJjdXJ2ZSI6MSwic3R5bGUiOnsiYm9keSI6eyJuYW1lIjoiZGFzaGVkIn19fV1d
        \[\begin{tikzcd}
            {a} && {a \times b} && {b} \\
            && {} \\
            {b} && {b \times a} && {a} \\
            \\
            \\
            \\
            && {}
            \arrow[from=1-3, to=1-1]
            \arrow[from=1-3, to=1-5]
            \arrow[from=3-3, to=3-1]
            \arrow[from=3-3, to=3-5]
            \arrow[from=1-1, to=3-5]
            \arrow[from=1-5, to=3-1]
            \arrow[from=1-3, to=3-3, curve={height=6pt}, dashed]
            \arrow[from=3-3, to=1-3, curve={height=6pt}, dashed]
        \end{tikzcd}\]
        and \cref{product has one identity} tells us that they are iso.
    \end{exercise}

    \begin{exercise}
        In this exercise we wish to show that products are associative up to isomorphism.
        So given $C$-objects $a, b, c$, form the products $(a \times b) \times c$ and $a \times (b \times c)$.
        Here they are with the relevant projection arrows:
        % https://q.uiver.app/?q=WzAsOCxbMSw3XSxbMSwwLCIoYSBcXHRpbWVzIGIpIl0sWzMsMCwiKGJcXHRpbWVzIGMpIl0sWzAsMSwiYSJdLFsyLDAsImIiXSxbNCwxLCJjIl0sWzMsMiwiYSBcXHRpbWVzIChiXFx0aW1lcyBjKSJdLFsxLDIsIihhIFxcdGltZXMgYikgXFx0aW1lcyBjIl0sWzEsM10sWzEsNF0sWzIsNF0sWzIsNV0sWzYsMl0sWzYsM10sWzcsMV0sWzcsNV1d
        \[\begin{tikzcd}
            & {(a \times b)} & {b} & {(b\times c)} \\
            {a} &&&& {c} \\
            & {(a \times b) \times c} && {a \times (b\times c)} \\
            \\
            \\
            \\
            \\
            & {}
            \arrow[from=1-2, to=2-1]
            \arrow[from=1-2, to=1-3]
            \arrow[from=1-4, to=1-3]
            \arrow[from=1-4, to=2-5]
            \arrow[from=3-4, to=1-4]
            \arrow[from=3-4, to=2-1]
            \arrow[from=3-2, to=1-2]
            \arrow[from=3-2, to=2-5]
        \end{tikzcd}\]
        We'll use the product definition side-by-side, noting that composition of projections gives us our arrows $a \times (b\times c) \to b$ and $(a\times b) \times c \to b$:
        % https://q.uiver.app/?q=WzAsOCxbMCwwXSxbMSwxLCJhIFxcdGltZXMgKGIgXFx0aW1lcyBjKSJdLFsxLDMsImEgXFx0aW1lcyBiIl0sWzAsMywiYSJdLFsyLDMsImIiXSxbMywxLCIoYSBcXHRpbWVzIGIpIFxcdGltZXMgYyJdLFs0LDMsImMiXSxbMywzLCJiXFx0aW1lcyBjIl0sWzEsMiwiIiwxLHsic3R5bGUiOnsiYm9keSI6eyJuYW1lIjoiZGFzaGVkIn19fV0sWzIsM10sWzIsNF0sWzEsM10sWzEsNCwiIiwxLHsic3R5bGUiOnsiYm9keSI6eyJuYW1lIjoiYmFycmVkIn19fV0sWzUsNCwiIiwxLHsic3R5bGUiOnsiYm9keSI6eyJuYW1lIjoiYmFycmVkIn19fV0sWzUsNl0sWzUsNywiIiwxLHsic3R5bGUiOnsiYm9keSI6eyJuYW1lIjoiZGFzaGVkIn19fV0sWzcsNF0sWzcsNl1d
        \[\begin{tikzcd}
            {} \\
            & {a \times (b \times c)} && {(a \times b) \times c} \\
            \\
            {a} & {a \times b} & {b} & {b\times c} & {c}
            \arrow[from=2-2, to=4-2, dashed]
            \arrow[from=4-2, to=4-1]
            \arrow[from=4-2, to=4-3]
            \arrow[from=2-2, to=4-1]
            \arrow["\shortmid" marking, from=2-2, to=4-3]
            \arrow["\shortmid" marking, from=2-4, to=4-3]
            \arrow[from=2-4, to=4-5]
            \arrow[from=2-4, to=4-4, dashed]
            \arrow[from=4-4, to=4-3]
            \arrow[from=4-4, to=4-5]
        \end{tikzcd}\]
        The definition yields unique arrows 
        $a \times (b\times c) \to b \times c$ and
        $(a\times b) \times c  \to a \times b$.
        Using these arrows along-side the `first-order' projection arrows, we use the product definition again to find the unique arrows
        % https://q.uiver.app/?q=WzAsNSxbMSwwXSxbMiwzLCJhIFxcdGltZXMgKGIgXFx0aW1lcyBjKSJdLFswLDQsImEgXFx0aW1lcyBiIl0sWzIsNSwiKGEgXFx0aW1lcyBiKSBcXHRpbWVzIGMiXSxbNCw0LCJiXFx0aW1lcyBjIl0sWzEsMiwiIiwxLHsic3R5bGUiOnsiYm9keSI6eyJuYW1lIjoiZGFzaGVkIn19fV0sWzMsNCwiIiwxLHsic3R5bGUiOnsiYm9keSI6eyJuYW1lIjoiZGFzaGVkIn19fV0sWzMsMl0sWzEsNF0sWzEsMywiIiwxLHsiY3VydmUiOi0yLCJzdHlsZSI6eyJib2R5Ijp7Im5hbWUiOiJkYXNoZWQifX19XSxbMywxLCIiLDEseyJjdXJ2ZSI6LTIsInN0eWxlIjp7ImJvZHkiOnsibmFtZSI6ImRhc2hlZCJ9fX1dXQ==
        \[\begin{tikzcd}
            & {} \\
            \\
            \\
            && {a \times (b \times c)} \\
            {a \times b} &&&& {b\times c} \\
            && {(a \times b) \times c}
            \arrow[from=4-3, to=5-1, dashed]
            \arrow[from=6-3, to=5-5, dashed]
            \arrow[from=6-3, to=5-1]
            \arrow[from=4-3, to=5-5]
            \arrow[from=4-3, to=6-3, curve={height=-12pt}, dashed]
            \arrow[from=6-3, to=4-3, curve={height=-12pt}, dashed]
        \end{tikzcd}\]
        Using \cref{product has one identity} gives that $(a \times b) \times c \cong a \times (b \times c)$.

    \end{exercise}

    \begin{exercise}
        \begin{enumerate}[(i)]
            Consider the situation of a pair of arrows to the codomain objects of the product arrow's constituent arrows:
            \item 
            % https://q.uiver.app/?q=WzAsOSxbMiw4XSxbNCwyLCJiIl0sWzQsNCwiZCJdLFswLDIsImEiXSxbMCw0LCJjIl0sWzIsMiwiYSBcXHRpbWVzIGIiXSxbMiw0LCJjIFxcdGltZXMgZCJdLFsyLDNdLFsyLDAsImUiXSxbMyw0LCJmIl0sWzEsMiwiZyJdLFs1LDMsInByX2EiXSxbNSwxLCJwcl9iIiwyXSxbNiw0XSxbNiwyXSxbNSw2LCJmXFx0aW1lcyBnIiwxLHsic3R5bGUiOnsiYm9keSI6eyJuYW1lIjoiZGFzaGVkIn19fV0sWzgsMywiaCIsMl0sWzgsMSwiayJdLFs4LDUsIlxcbGFuZ2xlIGgsIGsgXFxyYW5nbGUiLDEseyJzdHlsZSI6eyJib2R5Ijp7Im5hbWUiOiJkYXNoZWQifX19XV0=
            \[\begin{tikzcd}
                && {e} \\
                \\
                {a} && {a \times b} && {b} \\
                && {} \\
                {c} && {c \times d} && {d} \\
                \\
                \\
                \\
                && {}
                \arrow["{f}", from=3-1, to=5-1]
                \arrow["{g}", from=3-5, to=5-5]
                \arrow["{pr_a}", from=3-3, to=3-1]
                \arrow["{pr_b}"', from=3-3, to=3-5]
                \arrow[from=5-3, to=5-1]
                \arrow[from=5-3, to=5-5]
                \arrow["{f\times g}" description, from=3-3, to=5-3, dashed]
                \arrow["{h}"', from=1-3, to=3-1]
                \arrow["{k}", from=1-3, to=3-5]
                \arrow["{\langle h, k \rangle}" description, from=1-3, to=3-3, dashed]
            \end{tikzcd}\]
            We can use composition to collapse $a$ and $b$ out of the picture:
            % https://q.uiver.app/?q=WzAsNixbMiw2XSxbNCwyLCJkIl0sWzAsMiwiYyJdLFsyLDIsImMgXFx0aW1lcyBkIl0sWzIsMV0sWzIsMCwiZSJdLFszLDJdLFszLDFdLFs1LDMsIihmIFxcdGltZXMgZykgXFxjaXJjIFxcbGFuZ2xlIGgsIGsgXFxyYW5nbGUiLDFdLFs1LDIsImYgXFxjaXJjIGgiLDJdLFs1LDEsImcgXFxjaXJjIGsiXV0=
            \[\begin{tikzcd}
                && {e} \\
                && {} \\
                {c} && {c \times d} && {d} \\
                \\
                \\
                \\
                && {}
                \arrow[from=3-3, to=3-1]
                \arrow[from=3-3, to=3-5]
                \arrow["{(f \times g) \circ \langle h, k \rangle}" description, from=1-3, to=3-3]
                \arrow["{f \circ h}"', from=1-3, to=3-1]
                \arrow["{g \circ k}", from=1-3, to=3-5]
            \end{tikzcd}\]
            and thus find that $(f \times g) \circ \langle h, k \rangle$ is the unique arrow determined by $f \o h$ and $g \o k$.

            \item
            In the situation where we can place two product arrows end-to-end:
            % https://q.uiver.app/?q=WzAsMTEsWzIsOF0sWzQsMiwiYiJdLFs0LDQsImQiXSxbMCwyLCJhIl0sWzAsNCwiYyJdLFsyLDIsImEgXFx0aW1lcyBiIl0sWzIsNCwiYyBcXHRpbWVzIGQiXSxbMiwzXSxbMCwwLCJlIl0sWzQsMCwiZSciXSxbMiwwLCJlIFxcdGltZXMgZSciXSxbMyw0LCJmIl0sWzEsMiwiZyJdLFs1LDNdLFs1LDFdLFs2LDRdLFs2LDJdLFs1LDYsImZcXHRpbWVzIGciLDEseyJzdHlsZSI6eyJib2R5Ijp7Im5hbWUiOiJkYXNoZWQifX19XSxbOCwzLCJoIiwyXSxbMTAsOF0sWzEwLDldLFsxMCw1LCJoIFxcdGltZXMgayIsMSx7InN0eWxlIjp7ImJvZHkiOnsibmFtZSI6ImRhc2hlZCJ9fX1dLFs5LDEsImsiXV0=
            \[\begin{tikzcd}
                {e} && {e \times e'} && {e'} \\
                \\
                {a} && {a \times b} && {b} \\
                && {} \\
                {c} && {c \times d} && {d} \\
                \\
                \\
                \\
                && {}
                \arrow["{f}", from=3-1, to=5-1]
                \arrow["{g}", from=3-5, to=5-5]
                \arrow[from=3-3, to=3-1]
                \arrow[from=3-3, to=3-5]
                \arrow[from=5-3, to=5-1]
                \arrow[from=5-3, to=5-5]
                \arrow["{f\times g}" description, from=3-3, to=5-3, dashed]
                \arrow["{h}"', from=1-1, to=3-1]
                \arrow[from=1-3, to=1-1]
                \arrow[from=1-3, to=1-5]
                \arrow["{h \times k}" description, from=1-3, to=3-3, dashed]
                \arrow["{k}", from=1-5, to=3-5]
            \end{tikzcd}\]
            we again use composition to collapse the middle level
            % https://q.uiver.app/?q=WzAsOCxbMiw2XSxbNCwyLCJkIl0sWzAsMiwiYyJdLFsyLDIsImMgXFx0aW1lcyBkIl0sWzIsMV0sWzAsMCwiZSJdLFs0LDAsImUnIl0sWzIsMCwiZSBcXHRpbWVzIGUnIl0sWzMsMl0sWzMsMV0sWzcsNV0sWzcsNl0sWzcsMywiKGYgXFx0aW1lcyBnKSBcXGNpcmMoaFxcdGltZXMgaykiLDEseyJzdHlsZSI6eyJib2R5Ijp7Im5hbWUiOiJkYXNoZWQifX19XSxbNiwxLCJnIFxcY2lyYyBrIiwxXSxbNSwyLCJmIFxcY2lyYyBoIiwxXV0=
            \[\begin{tikzcd}
                {e} && {e \times e'} && {e'} \\
                && {} \\
                {c} && {c \times d} && {d} \\
                \\
                \\
                \\
                && {}
                \arrow[from=3-3, to=3-1]
                \arrow[from=3-3, to=3-5]
                \arrow[from=1-3, to=1-1]
                \arrow[from=1-3, to=1-5]
                \arrow["{(f \times g) \circ(h\times k)}" description, from=1-3, to=3-3, dashed]
                \arrow["{g \circ k}" description, from=1-5, to=3-5]
                \arrow["{f \circ h}" description, from=1-1, to=3-1]
            \end{tikzcd}\]
            Giving (more details needed?)
            $$(f \times g) \circ(h\times k) = (f \o g) \times (h \o k).$$
        \end{enumerate}
    \end{exercise}

    \subsection{Finite Products}
    \begin{defi}
        Given a $C$-object $a$, the \emph{finite product} (for some integer $m$) consists of object $a_m$, and $m$ projection arrows $pr_a^m$.
        % https://q.uiver.app/?q=WzAsNSxbMiw3XSxbMiwyXSxbMCwxXSxbMiwzLCJhXm0iXSxbMiwwLCJhIl0sWzMsNCwicHJfYV5tIiwxLHsiY3VydmUiOjV9XSxbMyw0LCJwcl9hXjEiLDEseyJjdXJ2ZSI6LTV9XSxbMyw0LCJwcl9hXjIiLDEseyJjdXJ2ZSI6LTN9XSxbNyw1LCIiLDEseyJsZW5ndGgiOjcwLCJsZXZlbCI6MSwic3R5bGUiOnsiYm9keSI6eyJuYW1lIjoiZG90dGVkIn0sImhlYWQiOnsibmFtZSI6Im5vbmUifX19XV0=
        \[\begin{tikzcd}
            && {a} \\
            {} \\
            && {} \\
            && {a^m} \\
            \\
            \\
            \\
            && {}
            \arrow["{pr_a^m}"{name=0, description}, from=4-3, to=1-3, curve={height=30pt}]
            \arrow["{pr_a^1}" description, from=4-3, to=1-3, curve={height=-30pt}]
            \arrow["{pr_a^2}"{name=1, description}, from=4-3, to=1-3, curve={height=-18pt}]
            \arrow[from=1, to=0, shorten <=7pt, shorten >=7pt, dotted, no head]
        \end{tikzcd}\]

        So that for any $m$ parallel arrows $f_i : c \to a$, there is a unique arrow $\langle f_1, \ldots, f_m \rangle : c \to a^m$ making
        % https://q.uiver.app/?q=WzAsNixbMiwxMF0sWzIsNV0sWzAsNF0sWzIsNiwiYV5tIl0sWzIsMywiYSJdLFsyLDAsImMiXSxbMyw0LCJwcl9hXm0iLDEseyJjdXJ2ZSI6MX1dLFszLDQsInByX2FeMSIsMSx7ImN1cnZlIjotNX1dLFszLDQsInByX2FeMiIsMSx7ImN1cnZlIjotM31dLFs1LDQsImZfMSIsMSx7ImN1cnZlIjo1fV0sWzUsNCwiZl8yIiwxLHsiY3VydmUiOjN9XSxbNSw0LCJmX20iLDEseyJjdXJ2ZSI6LTF9XSxbNSwzLCJcXGxhbmdsZSBmXzEsIFxcbGRvdHMsIGZfbSBcXHJhbmdsZSIsMSx7ImN1cnZlIjotNSwic3R5bGUiOnsiYm9keSI6eyJuYW1lIjoiZGFzaGVkIn19fV0sWzgsNiwiIiwxLHsibGVuZ3RoIjo3MCwibGV2ZWwiOjEsInN0eWxlIjp7ImJvZHkiOnsibmFtZSI6ImRvdHRlZCJ9LCJoZWFkIjp7Im5hbWUiOiJub25lIn19fV0sWzEwLDExLCIiLDEseyJsZW5ndGgiOjcwLCJsZXZlbCI6MSwic3R5bGUiOnsiYm9keSI6eyJuYW1lIjoiZG90dGVkIn0sImhlYWQiOnsibmFtZSI6Im5vbmUifX19XV0=
        \[\begin{tikzcd}
            && {c} \\
            \\
            \\
            && {a} \\
            {} \\
            && {} \\
            && {a^m} \\
            \\
            \\
            \\
            && {}
            \arrow["{pr_a^m}"{name=0, description}, from=7-3, to=4-3, curve={height=6pt}]
            \arrow["{pr_a^1}" description, from=7-3, to=4-3, curve={height=-30pt}]
            \arrow["{pr_a^2}"{name=1, description}, from=7-3, to=4-3, curve={height=-18pt}]
            \arrow["{f_1}" description, from=1-3, to=4-3, curve={height=30pt}]
            \arrow["{f_2}"{name=2, description}, from=1-3, to=4-3, curve={height=18pt}]
            \arrow["{f_m}"{name=3, description}, from=1-3, to=4-3, curve={height=-6pt}]
            \arrow["{\langle f_1, \ldots, f_m \rangle}" description, from=1-3, to=7-3, curve={height=-60pt}, dashed]
            \arrow[from=1, to=0, shorten <=4pt, shorten >=4pt, dotted, no head]
            \arrow[from=2, to=3, shorten <=4pt, shorten >=4pt, dotted, no head]
        \end{tikzcd}\]
        commute.
    \end{defi}

    \begin{defi}
        A more \emph{general finite product} of $m$ (not necessarily different) $C$-objects $a_1 \times a_2 \times \ldots \times a_m$
        consists of a $C$-object and $m$ projection arrows:
        % https://q.uiver.app/?q=WzAsNCxbMiwyLCJhXzEgXFx0aW1lcyBcXGxkb3RzIFxcdGltZXMgYV9tIl0sWzAsMCwiYV8xIl0sWzQsMCwiYV9tIl0sWzEsMCwiYV8yIl0sWzAsMSwicHJfe2FfMX0iXSxbMCwzLCJwcl97YV8yfSIsMl0sWzMsMiwiIiwyLHsic3R5bGUiOnsiYm9keSI6eyJuYW1lIjoiZG90dGVkIn0sImhlYWQiOnsibmFtZSI6Im5vbmUifX19XSxbMCwyLCJwcl97YV9tfSIsMl1d
        \[\begin{tikzcd}
            {a_1} & {a_2} &&& {a_m} \\
            \\
            && {a_1 \times \ldots \times a_m}
            \arrow["{pr_{a_1}}", from=3-3, to=1-1]
            \arrow["{pr_{a_2}}"', from=3-3, to=1-2]
            \arrow[from=1-2, to=1-5, dotted, no head]
            \arrow["{pr_{a_m}}"', from=3-3, to=1-5]
        \end{tikzcd}\]
        So that for any $c$-object with $m$ arrows $f_1 : c \to a_1$, $f_2 : c \to a_2$, etc,
        there is a unique arrow $\langle f_1, f_2, \ldots, f_m \rangle$ making
        % https://q.uiver.app/?q=WzAsNSxbMiw0LCJhXzEgXFx0aW1lcyBcXGxkb3RzIFxcdGltZXMgYV9tIl0sWzAsMiwiYV8xIl0sWzQsMiwiYV9tIl0sWzEsMiwiYV8yIl0sWzIsMCwiYyJdLFswLDEsInByX3thXzF9Il0sWzAsMywicHJfe2FfMn0iLDJdLFszLDIsIiIsMix7ImN1cnZlIjotMSwic3R5bGUiOnsiYm9keSI6eyJuYW1lIjoiZG90dGVkIn0sImhlYWQiOnsibmFtZSI6Im5vbmUifX19XSxbMCwyLCJwcl97YV9tfSIsMl0sWzQsMSwiZl8xIiwyXSxbNCwzLCJmXzIiXSxbNCwyLCJmX20iXSxbNCwwLCJcXGxhbmdsZSBmXzEsIGZfMiwgXFxsZG90cywgZl9tIFxccmFuZ2xlIiwwLHsic3R5bGUiOnsiYm9keSI6eyJuYW1lIjoiZGFzaGVkIn19fV1d
        \[\begin{tikzcd}
            && {c} \\
            \\
            {a_1} & {a_2} &&& {a_m} \\
            \\
            && {a_1 \times \ldots \times a_m}
            \arrow["{pr_{a_1}}", from=5-3, to=3-1]
            \arrow["{pr_{a_2}}"', from=5-3, to=3-2]
            \arrow[from=3-2, to=3-5, curve={height=-6pt}, dotted, no head]
            \arrow["{pr_{a_m}}"', from=5-3, to=3-5]
            \arrow["{f_1}"', from=1-3, to=3-1]
            \arrow["{f_2}", from=1-3, to=3-2]
            \arrow["{f_m}", from=1-3, to=3-5]
            \arrow["{\langle f_1, f_2, \ldots, f_m \rangle}", from=1-3, to=5-3, dashed]
        \end{tikzcd}\]
        commute.
    \end{defi}

    \begin{defi}
        A \emph{general product arrow} is given by a family of $m$ mappings between the components of two general products.

        % https://q.uiver.app/?q=WzAsOCxbMiw2LCJhXzEgXFx0aW1lcyBcXGxkb3RzIFxcdGltZXMgYV9tIl0sWzAsNCwiYV8xIl0sWzQsNCwiYV9tIl0sWzEsNCwiYV8yIl0sWzAsMiwiYl8xIl0sWzEsMiwiYl8yIl0sWzQsMiwiYl9tIl0sWzIsMCwiYl8xIFxcdGltZXMgYl8yIFxcdGltZXMgXFxsZG90cyBcXHRpbWVzIGJfbSJdLFswLDEsInByX3thXzF9Il0sWzAsMywicHJfe2FfMn0iLDJdLFszLDIsIiIsMix7InN0eWxlIjp7ImJvZHkiOnsibmFtZSI6ImRvdHRlZCJ9LCJoZWFkIjp7Im5hbWUiOiJub25lIn19fV0sWzAsMiwicHJfe2FfbX0iLDJdLFs1LDYsIiIsMix7InN0eWxlIjp7ImJvZHkiOnsibmFtZSI6ImRvdHRlZCJ9LCJoZWFkIjp7Im5hbWUiOiJub25lIn19fV0sWzcsNCwicHJfe2JfMX0iLDJdLFs3LDUsInByX3tiXzJ9Il0sWzcsNiwicHJfe2JfbX0iXSxbMSw0LCJmXzEiXSxbMyw1LCJmXzIiXSxbMiw2LCJmX20iLDJdLFswLDcsImZfMSBcXHRpbWVzIGZfMiBcXGNkb3RzIFxcdGltZXMgZl9tIiwyLHsic3R5bGUiOnsiYm9keSI6eyJuYW1lIjoiZGFzaGVkIn19fV0sWzE3LDE4LCIiLDIseyJvZmZzZXQiOi00LCJsZXZlbCI6MSwic3R5bGUiOnsiYm9keSI6eyJuYW1lIjoiZG90dGVkIn0sImhlYWQiOnsibmFtZSI6Im5vbmUifX19XV0=
        \[\begin{tikzcd}
            && {b_1 \times b_2 \times \ldots \times b_m} \\
            \\
            {b_1} & {b_2} &&& {b_m} \\
            \\
            {a_1} & {a_2} &&& {a_m} \\
            \\
            && {a_1 \times \ldots \times a_m}
            \arrow["{pr_{a_1}}", from=7-3, to=5-1]
            \arrow["{pr_{a_2}}"', from=7-3, to=5-2]
            \arrow[from=5-2, to=5-5, dotted, no head]
            \arrow["{pr_{a_m}}"', from=7-3, to=5-5]
            \arrow[from=3-2, to=3-5, dotted, no head]
            \arrow["{pr_{b_1}}"', from=1-3, to=3-1]
            \arrow["{pr_{b_2}}", from=1-3, to=3-2]
            \arrow["{pr_{b_m}}", from=1-3, to=3-5]
            \arrow["{f_1}", from=5-1, to=3-1]
            \arrow["{f_2}"{name=0}, from=5-2, to=3-2]
            \arrow["{f_m}"{name=1, swap}, from=5-5, to=3-5]
            \arrow["{f_1 \times f_2 \cdots \times f_m}"', from=7-3, to=1-3, dashed]
            \arrow[from=0, to=1, shift left=4, dotted, no head]
        \end{tikzcd}\]
        The product arrow $f_1 \times f_2 \cdots \times f_m$ is the unique arrow found by using 
        $$\langle f_1 \o pr_{a_1}, f_2 \o pr_{a_2}, \ldots, f_m \o pr_{a_m} \rangle$$
        in the general product definiton of $b_1 \times b_2 \times \ldots \times b_m$.
    \end{defi}

\section{Co-products}

    \begin{defi}
        A \emph{co-product} of $C$-objects $a$ and $b$ is given by a a $C$-object denoted $a+b$, and injection functions $i_a$ and $i_b$.
        % https://q.uiver.app/?q=WzAsNCxbNCwwLCJiIl0sWzEsMF0sWzAsMCwiYSJdLFsyLDAsImErYiJdLFsyLDMsImlfYSIsMl0sWzAsMywiaV9iIl1d
        \[\begin{tikzcd}
            {a} & {} & {a+b} && {b}
            \arrow["{i_a}"', from=1-1, to=1-3]
            \arrow["{i_b}", from=1-5, to=1-3]
        \end{tikzcd}\]

        For any $f : a \to c$ and $g : b \to c$, there is a unique arrow
        $[f,g] : (a + b) \to c$ so that
        % https://q.uiver.app/?q=WzAsNSxbNCwwLCJiIl0sWzEsMF0sWzAsMCwiYSJdLFsyLDAsImErYiJdLFsyLDIsImMiXSxbMiwzLCJpX2EiLDJdLFswLDMsImlfYiJdLFsyLDQsImYiLDJdLFswLDQsImciXSxbMyw0LCJbZixnXSIsMSx7InN0eWxlIjp7ImJvZHkiOnsibmFtZSI6ImRhc2hlZCJ9fX1dXQ==
        \[\begin{tikzcd}
            {a} & {} & {a+b} && {b} \\
            \\
            && {c}
            \arrow["{i_a}"', from=1-1, to=1-3]
            \arrow["{i_b}", from=1-5, to=1-3]
            \arrow["{f}"', from=1-1, to=3-3]
            \arrow["{g}", from=1-5, to=3-3]
            \arrow["{[f,g]}" description, from=1-3, to=3-3, dashed]
        \end{tikzcd}\]
        commutes.
    \end{defi}

    \begin{exercise}
        In \textbf{Set} we are told that the co-product $A+B$ is the disjoint union, with $i_A$ and $i_B$ being the disjoint identity function (Ie, $i_A(x) = (x, 0)$ for $x \in A$, and $i_B(x) = (x, 1)$ for $x \in B$.
        Now suppose we have functions $f : A \to C$ and $g : B \to C$.
        % https://q.uiver.app/?q=WzAsNSxbNCwwLCJCIl0sWzEsMF0sWzAsMCwiQSJdLFsyLDAsImErYiJdLFsyLDIsIkMiXSxbMiwzLCJpX0EiLDJdLFswLDMsImlfQiJdLFsyLDQsImYiLDJdLFswLDQsImciXSxbMyw0LCJbZixnXSIsMSx7InN0eWxlIjp7ImJvZHkiOnsibmFtZSI6ImRhc2hlZCJ9fX1dXQ==
        \[\begin{tikzcd}
            {A} & {} & {A+B} && {B} \\
            \\
            && {C}
            \arrow["{i_A}"', from=1-1, to=1-3]
            \arrow["{i_B}", from=1-5, to=1-3]
            \arrow["{f}"', from=1-1, to=3-3]
            \arrow["{g}", from=1-5, to=3-3]
            \arrow["{[f,g]}" description, from=1-3, to=3-3, dashed]
        \end{tikzcd}\]
        We can find $[f,g] : A + B \to C$ making this diagram commute by using the rule
            \[ [f,g](\langle x, y \rangle ) =\begin{cases} 
                  f(x) & y = 0 \\
                  g(x) & y = 1. 
               \end{cases}
            \]
        To see that it $[f,g]$ is unique, notice that as given, $[f,g] \o i_A = f$ iff $[f,g](i_A(x)) = f(x)$ for all $x \in A$. Similarly on the $B$ side - there is no other way to recover the action of $f$ and $g$ out of $A+B$.
    \end{exercise}

    \begin{exercise}
        If $A \cup B = \emptyset$ then we notice that $A\cup B$ satisfies the definition of co-product
        % https://q.uiver.app/?q=WzAsNSxbNCwwLCJCIl0sWzEsMF0sWzAsMCwiQSJdLFsyLDAsIkFcXGN1cCBCIl0sWzIsMiwiQyJdLFsyLDMsImlfQSIsMl0sWzAsMywiaV9CIl0sWzIsNCwiZiIsMl0sWzAsNCwiZyJdLFszLDQsIltmLGddIiwxLHsic3R5bGUiOnsiYm9keSI6eyJuYW1lIjoiZGFzaGVkIn19fV1d
        \[\begin{tikzcd}
            {A} & {} & {A\cup B} && {B} \\
            \\
            && {C}
            \arrow["{i_A}"', from=1-1, to=1-3]
            \arrow["{i_B}", from=1-5, to=1-3]
            \arrow["{f}"', from=1-1, to=3-3]
            \arrow["{g}", from=1-5, to=3-3]
            \arrow["{[f,g]}" description, from=1-3, to=3-3, dashed]
        \end{tikzcd}\]
        where $i_A$ and $i_B$ are the inclusion functions, and 
            \[ [f,g]=\begin{cases} 
                  f(x) & x \in A \\
                  g(x) & x \in B,
               \end{cases}
            \]
        which is well defined because any $x$ is in either $A$ or $B$ but not both.
        Then applying the dual of \cref{products are isomorphic}, co-products are isomorphic and therefore $A \cup B \cong A + B$.
    \end{exercise}

    \begin{defi}
        Given two co-products $a + b$ and $c + d$, and arrows $f : a \to c$ and $g : b \to d$ 
        % https://q.uiver.app/?q=WzAsNyxbNCwwLCJiIl0sWzEsMF0sWzAsMCwiYSJdLFsyLDAsImEgKyBiIl0sWzIsMiwiYyArIGQiXSxbMCwyLCJjIl0sWzQsMiwiZCJdLFsyLDMsImlfYSIsMl0sWzAsMywiaV9iIl0sWzYsNCwiaV9kIl0sWzUsNCwiaV9jIiwyXSxbMiw1LCJmIiwxXSxbMCw2LCJnIiwxXV0=
        \[\begin{tikzcd}
            {a} & {} & {a + b} && {b} \\
            \\
            {c} && {c + d} && {d}
            \arrow["{i_a}"', from=1-1, to=1-3]
            \arrow["{i_b}", from=1-5, to=1-3]
            \arrow["{i_d}", from=3-5, to=3-3]
            \arrow["{i_c}"', from=3-1, to=3-3]
            \arrow["{f}" description, from=1-1, to=3-1]
            \arrow["{g}" description, from=1-5, to=3-5]
        \end{tikzcd}\]
        the \emph{co-product arrow} $f + g$ is found by using $[i_c \o f, i_d \o g]$ in the co-product definition of $c+d$.
        % https://q.uiver.app/?q=WzAsNyxbNCwwLCJiIl0sWzEsMF0sWzAsMCwiYSJdLFsyLDAsImEgKyBiIl0sWzIsMiwiYyArIGQiXSxbMCwyLCJjIl0sWzQsMiwiZCJdLFsyLDMsImlfYSIsMl0sWzAsMywiaV9iIl0sWzYsNCwiaV9kIl0sWzUsNCwiaV9jIiwyXSxbMiw1LCJmIiwxXSxbMCw2LCJnIiwxXSxbMyw0LCJmK2ciLDEseyJzdHlsZSI6eyJib2R5Ijp7Im5hbWUiOiJkYXNoZWQifX19XV0=
        \[\begin{tikzcd}
            {a} & {} & {a + b} && {b} \\
            \\
            {c} && {c + d} && {d}
            \arrow["{i_a}"', from=1-1, to=1-3]
            \arrow["{i_b}", from=1-5, to=1-3]
            \arrow["{i_d}", from=3-5, to=3-3]
            \arrow["{i_c}"', from=3-1, to=3-3]
            \arrow["{f}" description, from=1-1, to=3-1]
            \arrow["{g}" description, from=1-5, to=3-5]
            \arrow["{f+g}" description, from=1-3, to=3-3, dashed]
        \end{tikzcd}\]
    \end{defi}

\section{Equalizers}

    \begin{defi} \label{equalizer}
        An arrow $i$ \emph{equalizes} $f$ and $g$ if they are laid out as follows
        % https://q.uiver.app/?q=WzAsNCxbMSwwXSxbMiwxLCJhIl0sWzQsMSwiYiJdLFswLDEsImUiXSxbMSwyLCJmIl0sWzEsMiwiZyIsMix7Im9mZnNldCI6MX1dLFszLDEsImkiLDJdXQ==
        \[\begin{tikzcd}
            & {} \\
            {e} && {a} && {b}
            \arrow["{f}", from=2-3, to=2-5]
            \arrow["{g}"', from=2-3, to=2-5, shift right=1]
            \arrow["{i}"', from=2-1, to=2-3]
        \end{tikzcd}\]
        where $f \o i = g \o i$. Additionaly we demand the limiting property: if another $e^*$ and $i^*$ work as above, there is a unique arrow $e^* \to e$ making
        % https://q.uiver.app/?q=WzAsNSxbMSwwXSxbMiwxLCJhIl0sWzQsMSwiYiJdLFswLDEsImUiXSxbMCwzLCJlXioiXSxbMSwyLCJmIl0sWzEsMiwiZyIsMix7Im9mZnNldCI6MX1dLFszLDEsImkiLDJdLFs0LDEsImleKiIsMl0sWzQsMywiIiwxLHsic3R5bGUiOnsiYm9keSI6eyJuYW1lIjoiZGFzaGVkIn19fV1d
        \[\begin{tikzcd}
            & {} \\
            {e} && {a} && {b} \\
            \\
            {e^*}
            \arrow["{f}", from=2-3, to=2-5]
            \arrow["{g}"', from=2-3, to=2-5, shift right=1]
            \arrow["{i}"', from=2-1, to=2-3]
            \arrow["{i^*}"', from=4-1, to=2-3]
            \arrow[from=4-1, to=2-1, dashed]
        \end{tikzcd}\] 
        commute.
    \end{defi}

    \begin{fact}
        Every equalizer is monic.
    \end{fact}

    \begin{fact}
        An epic equalizer is iso.
        % https://q.uiver.app/?q=WzAsNSxbMSwwXSxbMiwxLCJhIl0sWzQsMSwiYiJdLFswLDEsImUiXSxbMCwzLCJhIl0sWzEsMiwiZiJdLFsxLDIsImciLDIseyJvZmZzZXQiOjF9XSxbMywxLCJpIiwyXSxbNCwxLCIxX2EiLDJdLFs0LDMsIiIsMSx7InN0eWxlIjp7ImJvZHkiOnsibmFtZSI6ImRhc2hlZCJ9fX1dXQ==
        \[\begin{tikzcd}
            & {} \\
            {e} && {a} && {b} \\
            \\
            {a}
            \arrow["{f}", from=2-3, to=2-5]
            \arrow["{g}"', from=2-3, to=2-5, shift right=1]
            \arrow["{i}"', from=2-1, to=2-3]
            \arrow["{1_a}"', from=4-1, to=2-3]
            \arrow[from=4-1, to=2-1, dashed]
        \end{tikzcd}\]
    \end{fact}

    \begin{exercise}
        Working in \textbf{Set}, we wish to show that monics are equalizers.
        Suppose we have some injective function $i : E \to A$:
        % https://q.uiver.app/?q=WzAsMyxbMSwwXSxbMiwxLCJBIl0sWzAsMSwiRSJdLFsyLDEsImkiLDAseyJzdHlsZSI6eyJ0YWlsIjp7Im5hbWUiOiJtb25vIn19fV1d
        \[\begin{tikzcd}
            & {} \\
            {E} && {A}
            \arrow["{i}", from=2-1, to=2-3, tail]
        \end{tikzcd}\]
        We seek functions which $i$ is an equalizer.
        Let $f, g : A \too \set{0,1}$ be given by
        $$f(x)  = 1$$
        \[ g(x)=\begin{cases} 
              1 & x \in i(E) \\
              0 & x \notin i(E). 
           \end{cases}
        \]
        Now clearly $g(i(x)) = f(i(x))$ for all $x \in E$, so $i \o f = i \o g$.
        Supposing that there is another $i^* : C \to A$ such that $i^* \o f = i^* \o g$.
        We must have ...?
        TODO: Finish showing universal property
        % https://q.uiver.app/?q=WzAsNSxbMSwwXSxbMiwxLCJBIl0sWzAsMSwiRSJdLFs0LDEsIlxcezAsMVxcfSJdLFswLDMsIkMiXSxbMiwxLCJpIiwwLHsic3R5bGUiOnsidGFpbCI6eyJuYW1lIjoibW9ubyJ9fX1dLFsxLDMsImciXSxbMSwzLCJmIiwyLHsib2Zmc2V0IjoxfV0sWzQsMSwiaV4qIiwyXSxbNCwyLCIiLDAseyJzdHlsZSI6eyJib2R5Ijp7Im5hbWUiOiJkYXNoZWQifX19XV0=
        \[\begin{tikzcd}
            & {} \\
            {E} && {A} && {\{0,1\}} \\
            \\
            {C}
            \arrow["{i}", from=2-1, to=2-3, tail]
            \arrow["{g}", from=2-3, to=2-5]
            \arrow["{f}"', from=2-3, to=2-5, shift right=1]
            \arrow["{i^*}"', from=4-1, to=2-3]
            \arrow[from=4-1, to=2-1, dashed]
        \end{tikzcd}\]
    \end{exercise}

    \begin{exercise}
        Working in a poset, suppose that $i$ equalizes $f$ and $g$. Recall that any 2 parallel arrows are equal. Then in particular $f \o i = g \o i$ as follows:
        % https://q.uiver.app/?q=WzAsNCxbMSwwXSxbMiwxLCJhIl0sWzQsMSwiYiJdLFswLDEsImUiXSxbMSwyLCJmIl0sWzEsMiwiZyIsMix7Im9mZnNldCI6MX1dLFszLDEsImkiLDJdXQ==
        \[\begin{tikzcd}
            & {} \\
            {e} && {a} && {b}
            \arrow["{f}", from=2-3, to=2-5]
            \arrow["{g}"', from=2-3, to=2-5, shift right=1]
            \arrow["{i}"', from=2-1, to=2-3]
        \end{tikzcd}\]
        But plugging $a$ and $1_a$ into the definition for equalizer, since $f \o 1_a = f = g = g \o 1_a$ we retrieve the unique arrow
        $a \to e$.
        % https://q.uiver.app/?q=WzAsNSxbMSwwXSxbMiwxLCJhIl0sWzQsMSwiYiJdLFswLDEsImUiXSxbMCwzLCJhIl0sWzEsMiwiZiJdLFsxLDIsImciLDIseyJvZmZzZXQiOjF9XSxbMywxLCJpIiwyXSxbNCwzLCIiLDIseyJzdHlsZSI6eyJib2R5Ijp7Im5hbWUiOiJkYXNoZWQifX19XSxbNCwxLCIxX2EiLDJdXQ==
        \[\begin{tikzcd}
            & {} \\
            {e} && {a} && {b} \\
            \\
            {a}
            \arrow["{f}", from=2-3, to=2-5]
            \arrow["{g}"', from=2-3, to=2-5, shift right=1]
            \arrow["{i}"', from=2-1, to=2-3]
            \arrow[from=4-1, to=2-1, dashed]
            \arrow["{1_a}"', from=4-1, to=2-3]
        \end{tikzcd}\]
        Since we have arrows $a \to e$ and $e \to a$ we simply apply the antisymmetric property of posets to determine $e = a$, and the single-arrow property to determine that $i = 1_a$.

        
    \end{exercise}

\section{Limits and co-limits}

    \begin{defi}
        A \emph{diagram} is informally a `view' of some objects and arrows within a category.
        We haven't looked at functors yet but formally, a diagram in a category $C$ is a functor $F : J \to C$ where $J$ is an indexing category. (Riehl pg 38)
    \end{defi}

    \begin{defi}
        Given a diagram $D$, a a \emph{D-cone} consists of a $C$-object $c$ together with component arrows $f_i : c_i \to d_i$ for each $d_i \in d$ that commute with any arrow $g$ in $D$.
        % https://q.uiver.app/?q=WzAsNCxbMSwwXSxbMCwzLCJkX2kiXSxbMiwzLCJkX2oiXSxbMSwxLCJjIl0sWzEsMl0sWzMsMSwiZl9pIiwyXSxbMywyLCJmX2oiXV0=
        \[\begin{tikzcd}
            & {} \\
            & {c} \\
            \\
            {d_i} && {d_j}
            \arrow[from=4-1, to=4-3]
            \arrow["{f_i}"', from=2-2, to=4-1]
            \arrow["{f_j}", from=2-2, to=4-3]
        \end{tikzcd}\]
    \end{defi}

    \begin{defi}
        Given a diagram $D$, a \emph{limit} for $D$ is a $D$-cone such that any other $D$-cone factors through uniquely.
        % https://q.uiver.app/?q=WzAsNSxbMiwwXSxbMSwzLCJkX2kiXSxbMywzLCJkX2oiXSxbMCwxLCJjIl0sWzQsMSwiYyciXSxbMSwyXSxbMywxLCJmX2kiLDJdLFszLDIsImZfaiJdLFs0LDEsImZfaSciLDJdLFs0LDIsImZfaiciLDJdLFs0LDMsIiIsMix7InN0eWxlIjp7ImJvZHkiOnsibmFtZSI6ImRhc2hlZCJ9fX1dXQ==
        \[\begin{tikzcd}
            && {} \\
            {c} &&&& {c'} \\
            \\
            & {d_i} && {d_j}
            \arrow[from=4-2, to=4-4]
            \arrow["{f_i}"', from=2-1, to=4-2]
            \arrow["{f_j}", from=2-1, to=4-4]
            \arrow["{f_i'}"', from=2-5, to=4-2]
            \arrow["{f_j'}"', from=2-5, to=4-4]
            \arrow[from=2-5, to=2-1, dashed]
        \end{tikzcd}\]
    \end{defi}

    We can apply cones to define some earlier objects in a more succinct way.

    \begin{defi}
        The \emph{product} of $a$ and $b$ is a limit for the arrow-less diagram
        % https://q.uiver.app/?q=WzAsMixbMCwwLCJhIl0sWzIsMCwiYiJdXQ==
        \[\begin{tikzcd}
            {a} && {b.}
        \end{tikzcd}\]
    \end{defi}

    \begin{defi}
        An \emph{equalizer} of $f$ and $g$ is a limit for the diagram 
        % https://q.uiver.app/?q=WzAsMixbMCwwLCJhIl0sWzIsMCwiYiJdLFswLDEsImciLDIseyJvZmZzZXQiOjF9XSxbMCwxLCJmIiwwLHsib2Zmc2V0IjotMX1dXQ==
        \[\begin{tikzcd}
            {a} && {b}
            \arrow["{g}"', from=1-1, to=1-3, shift right=1]
            \arrow["{f}", from=1-1, to=1-3, shift left=1]
        \end{tikzcd}\]
    \end{defi}

    \begin{defi}
        A \emph{terminal object} is a limit for the empty diagram.
    \end{defi}

    \begin{exercise}
        Dualize to reach definitions for \emph{co-cones} and \emph{co-limits}.
    \end{exercise}

\section{Co-equalizers}

    \begin{defi}
        A \emph{co-equalizer} of $f, g : a \too b$ is a co-limit for the diagram
        % https://q.uiver.app/?q=WzAsMixbMCwwLCJhIl0sWzIsMCwiYiJdLFswLDEsImciLDIseyJvZmZzZXQiOjF9XSxbMCwxLCJmIiwwLHsib2Zmc2V0IjotMX1dXQ==
        \[\begin{tikzcd}
            {a} && {b}
            \arrow["{g}"', from=1-1, to=1-3, shift right=1]
            \arrow["{f}", from=1-1, to=1-3, shift left=1]
        \end{tikzcd}\]
    \end{defi}
    Let's work backwards from that definition and try to reach something similar to \cref{equalizer} for equalizers!
    A co-cone over the above diagram consist of an object $c$, and arrows $f_a : a \to c$ and $f_b : b \to c$ such that
    % https://q.uiver.app/?q=WzAsMyxbMCwwLCJhIl0sWzIsMCwiYiJdLFsxLDIsImMiXSxbMCwxLCJnIiwyLHsib2Zmc2V0IjoxfV0sWzAsMSwiZiIsMCx7Im9mZnNldCI6LTF9XSxbMCwyLCJmX2EiLDJdLFsxLDIsImZfYiJdXQ==
    \[\begin{tikzcd}
        {a} && {b} \\
        \\
        & {c}
        \arrow["{g}"', from=1-1, to=1-3, shift right=1]
        \arrow["{f}", from=1-1, to=1-3, shift left=1]
        \arrow["{f_a}"', from=1-1, to=3-2]
        \arrow["{f_b}", from=1-3, to=3-2]
    \end{tikzcd}\]
    commutes. But that means that $f_b \o f = f_a$ and $f_b \o g = f_a$. So we can simply drop the $f_a$ and demand an $f_b$ making
    % https://q.uiver.app/?q=WzAsMyxbMCwwLCJhIl0sWzIsMCwiYiJdLFs0LDAsImMiXSxbMCwxLCJnIiwyLHsib2Zmc2V0IjoxfV0sWzAsMSwiZiIsMCx7Im9mZnNldCI6LTF9XSxbMSwyLCJmX2IiXV0=
    \[\begin{tikzcd}
        {a} && {b} && {c}
        \arrow["{g}"', from=1-1, to=1-3, shift right=1]
        \arrow["{f}", from=1-1, to=1-3, shift left=1]
        \arrow["{f_b}", from=1-3, to=1-5]
    \end{tikzcd}\]
    commute.
    That gave us our co-cone, but what we want is a co-limit: \emph{the} co-cone that is universal among co-cones.
    So if we have a $c'$ and $f_b'$ such that $f_b' \o f = f_b' \o g$, then $c$ must must factor through $c'$ in a unique way:
    % https://q.uiver.app/?q=WzAsNCxbMCwwLCJhIl0sWzIsMCwiYiJdLFs0LDAsImMiXSxbNCwyLCJjJyJdLFswLDEsImciLDIseyJvZmZzZXQiOjF9XSxbMCwxLCJmIiwwLHsib2Zmc2V0IjotMX1dLFsxLDIsImZfYiJdLFsxLDMsImZfYiciLDJdLFsyLDMsIiIsMCx7InN0eWxlIjp7ImJvZHkiOnsibmFtZSI6ImRhc2hlZCJ9fX1dXQ==
    \[\begin{tikzcd}
        {a} && {b} && {c} \\
        \\
        &&&& {c'}
        \arrow["{g}"', from=1-1, to=1-3, shift right=1]
        \arrow["{f}", from=1-1, to=1-3, shift left=1]
        \arrow["{f_b}", from=1-3, to=1-5]
        \arrow["{f_b'}"', from=1-3, to=3-5]
        \arrow[from=1-5, to=3-5, dashed]
    \end{tikzcd}\]

\section{Pullbacks}
    \begin{defi}
        Given $f: a \to c$ and $g : b \to c$, a \emph{pullback} is a limit for the diagram 
        % https://q.uiver.app/?q=WzAsMyxbMCwwLCJhIl0sWzIsMCwiYyJdLFs0LDAsImIiXSxbMCwxLCJmIl0sWzIsMSwiZyIsMl1d
        \[\begin{tikzcd}
            {a} && {c} && {b}
            \arrow["{f}", from=1-1, to=1-3]
            \arrow["{g}"', from=1-5, to=1-3]
        \end{tikzcd}\]
    \end{defi}
    A cone for this diagram takes the form of an object $p$ with arrows $f' : p \to b$ and $g' : p \to a$ so that
    % https://q.uiver.app/?q=WzAsNCxbMCwyLCJhIl0sWzIsMiwiYyJdLFsyLDAsImIiXSxbMCwwLCJwIl0sWzAsMSwiZiJdLFsyLDEsImciLDJdLFszLDIsImYnIiwyXSxbMywwLCJnJyJdXQ==
    \[\begin{tikzcd}
        {p} && {b} \\
        \\
        {a} && {c}
        \arrow["{f}", from=3-1, to=3-3]
        \arrow["{g}"', from=1-3, to=3-3]
        \arrow["{f'}"', from=1-1, to=1-3]
        \arrow["{g'}", from=1-1, to=3-1]
    \end{tikzcd}\]
    commutes.
    And the limit means that any cone factors through:
    % https://q.uiver.app/?q=WzAsNSxbMSwzLCJhIl0sWzMsMywiYyJdLFszLDEsImIiXSxbMSwxLCJwIl0sWzAsMCwicCciXSxbMCwxLCJmIl0sWzIsMSwiZyIsMl0sWzMsMiwiZiciLDJdLFszLDAsImcnIl0sWzQsMywiIiwwLHsic3R5bGUiOnsiYm9keSI6eyJuYW1lIjoiZGFzaGVkIn19fV0sWzQsMiwiZicnIl0sWzQsMCwiZycnIl1d
    \[\begin{tikzcd}
        {p'} \\
        & {p} && {b} \\
        \\
        & {a} && {c}
        \arrow["{f}", from=4-2, to=4-4]
        \arrow["{g}"', from=2-4, to=4-4]
        \arrow["{f'}"', from=2-2, to=2-4]
        \arrow["{g'}", from=2-2, to=4-2]
        \arrow[from=1-1, to=2-2, dashed]
        \arrow["{f''}", from=1-1, to=2-4]
        \arrow["{g''}", from=1-1, to=4-2]
    \end{tikzcd}\]

    \begin{exercise}
        Suppose
        % https://q.uiver.app/?q=WzAsNCxbMCwyLCJjIl0sWzAsMCwiYSJdLFsyLDAsImIiXSxbMiwyLCJkIl0sWzEsMiwiZyJdLFswLDMsImYiXSxbMiwzXSxbMSwwXV0=
        \[\begin{tikzcd}
            {a} && {b} \\
            \\
            {c} && {d}
            \arrow["{g}", from=1-1, to=1-3]
            \arrow["{f}", from=3-1, to=3-3]
            \arrow[from=1-3, to=3-3]
            \arrow[from=1-1, to=3-1]
        \end{tikzcd}\]
        is a pullback square, and $f$ is monic.
        Now suppose we have parallel $h_1, h_2 : e \too a$
        % https://q.uiver.app/?q=WzAsNSxbMiwyLCJjIl0sWzIsMCwiYSJdLFs0LDAsImIiXSxbNCwyLCJkIl0sWzAsMCwiZSJdLFsxLDIsImciXSxbMCwzLCJmIl0sWzIsMywiZyciLDFdLFsxLDAsImYnIiwxXSxbNCwxLCJoXzEiLDEseyJjdXJ2ZSI6LTF9XSxbNCwxLCJoXzIiLDEseyJjdXJ2ZSI6MX1dXQ==
        \[\begin{tikzcd}
            {e} && {a} && {b} \\
            \\
            && {c} && {d}
            \arrow["{g}", from=1-3, to=1-5]
            \arrow["{f}", from=3-3, to=3-5]
            \arrow["{g'}" description, from=1-5, to=3-5]
            \arrow["{f'}" description, from=1-3, to=3-3]
            \arrow["{h_1}" description, from=1-1, to=1-3, curve={height=-6pt}]
            \arrow["{h_2}" description, from=1-1, to=1-3, curve={height=6pt}]
        \end{tikzcd}\]
        so that $g \o h_1 = g \o h_2$.
        Then $$g' \o g \o h_1 = g' \o g \o h_2.$$
        From the original pullback we have $g' \o g = f \o f'$,
        so substituting on both sides gives $f \o f' \o h_1 = f \o f' \o h_2$.
        Since $f$ is monic, $f' \o h_1 = f' \o h_2$.

        But also, substituting on one side gives $g' \o g \o h_1 = f \o f' \o h_1$.
        Then
        % https://q.uiver.app/?q=WzAsNCxbMCwyLCJjIl0sWzAsMCwiZSJdLFsyLDAsImIiXSxbMiwyLCJkIl0sWzEsMiwiZyBcXG8gaF8xID0gZyBcXG8gaF8yIl0sWzAsMywiZiJdLFsyLDMsImcnIiwxXSxbMSwwLCJmJyBcXG8gaF8xID0gZicgXFxvIGhfMiIsMV1d
        \[\begin{tikzcd}
            {e} && {b} \\
            \\
            {c} && {d}
            \arrow["{g \o h_1 = g \o h_2}", from=1-1, to=1-3]
            \arrow["{f}", from=3-1, to=3-3]
            \arrow["{g'}" description, from=1-3, to=3-3]
            \arrow["{f' \o h_1 = f' \o h_2}" description, from=1-1, to=3-1]
        \end{tikzcd}\]
        commutes, meaning it must factor uniquely through our original pullback.
        Thus the arrow $e \to a$ is unique, meaning $h_1 = h_2$.
        Conclude $f$ is monic.

    \end{exercise}

\section{Pushouts}

    \begin{defi}
        Given arrows $f : a \to b$ and $g : a \to c$, a \emph{pushout} is a colimit for the diagram
        % https://q.uiver.app/?q=WzAsMyxbMiwwLCJhIl0sWzAsMCwiYiJdLFs0LDAsImMiXSxbMCwxLCJmIl0sWzAsMiwiZyIsMl1d
        \[\begin{tikzcd}
            {b} && {a} && {c}
            \arrow["{f}", from=1-3, to=1-1]
            \arrow["{g}"', from=1-3, to=1-5]
        \end{tikzcd}\]
    \end{defi}

    The co-limit, or universal co-cone for the above diagram consists of an object $d$ and arrows $f' : c \to d$ and $g' : b \to d$ so that
    $g' \o f = f' \o g$.
    Further, if there is a $d', f'', g''$ behaving the same way, then $d$ will factor uniquely through it.
    % https://q.uiver.app/?q=WzAsNSxbMCwwLCJhIl0sWzIsMCwiYiJdLFswLDIsImMiXSxbMiwyLCJkIl0sWzQsMywiZCciXSxbMCwxLCJmIl0sWzAsMiwiZyIsMl0sWzIsMywiZiciXSxbMSwzLCJnJyIsMl0sWzEsNCwiZycnIl0sWzIsNCwiZicnIiwyXSxbMyw0LCIiLDIseyJzdHlsZSI6eyJib2R5Ijp7Im5hbWUiOiJkYXNoZWQifX19XV0=
    \[\begin{tikzcd}
        {a} && {b} \\
        \\
        {c} && {d} \\
        &&&& {d'}
        \arrow["{f}", from=1-1, to=1-3]
        \arrow["{g}"', from=1-1, to=3-1]
        \arrow["{f'}", from=3-1, to=3-3]
        \arrow["{g'}"', from=1-3, to=3-3]
        \arrow["{g''}", from=1-3, to=4-5]
        \arrow["{f''}"', from=3-1, to=4-5]
        \arrow[from=3-3, to=4-5, dashed]
    \end{tikzcd}\]

\section{Completeness}
    \begin{defi}
        A category is \emph{complete} if every diagram has a limit.
    \end{defi}
    \begin{defi}
        A category if \emph{finitely complete} if every \emph{finite} diagram has a limit.
    \end{defi}
    \begin{fact}
        If $C$ has a terminal object and a pullback for each pair of arrows with common codomain,
        then $C$ is finitely complete.
    \end{fact}

    \begin{exercise}
        Working in a category with pullbacks and products, we demonstrate how to constuct equalizers.
        Given a parallel pair $f, g : a \too b$, 
        first form the product:
        % https://q.uiver.app/?q=WzAsMyxbMCwwLCJhIl0sWzQsMCwiYiJdLFsyLDIsImEgXFx0aW1lcyBiIl0sWzIsMF0sWzIsMV0sWzAsMSwiZiIsMCx7Im9mZnNldCI6LTF9XSxbMCwxLCJnIiwyLHsib2Zmc2V0IjoxfV1d
        \[\begin{tikzcd}
            {a} &&&& {b} \\
            \\
            && {a \times b}
            \arrow[from=3-3, to=1-1]
            \arrow[from=3-3, to=1-5]
            \arrow["{f}", from=1-1, to=1-5, shift left=1]
            \arrow["{g}"', from=1-1, to=1-5, shift right=1]
        \end{tikzcd}\]
        Then we can find the two product arrows $f' = \langle 1_a, f \rangle$ and $g' = \langle 1_a, g \rangle$.
        % https://q.uiver.app/?q=WzAsNCxbMiwwLCJhIl0sWzQsMiwiYiJdLFsyLDIsImEgXFx0aW1lcyBiIl0sWzAsMiwiYSJdLFswLDEsImYiLDAseyJvZmZzZXQiOi0xfV0sWzAsMSwiZyIsMix7Im9mZnNldCI6MX1dLFsyLDMsInByX2EiXSxbMCwzLCIxX2EiLDJdLFswLDIsIlxcbGFuZ2xlMV9hLCBmIFxccmFuZ2xlIiwyLHsib2Zmc2V0IjoxLCJzdHlsZSI6eyJib2R5Ijp7Im5hbWUiOiJkYXNoZWQifX19XSxbMCwyLCJcXGxhbmdsZTFfYSwgZyBcXHJhbmdsZSIsMCx7Im9mZnNldCI6LTEsInN0eWxlIjp7ImJvZHkiOnsibmFtZSI6ImRhc2hlZCJ9fX1dLFsyLDEsInByX2IiLDJdXQ==
        \[\begin{tikzcd}
            && {a} \\
            \\
            {a} && {a \times b} && {b}
            \arrow["{f}", from=1-3, to=3-5, shift left=1]
            \arrow["{g}"', from=1-3, to=3-5, shift right=1]
            \arrow["{pr_a}", from=3-3, to=3-1]
            \arrow["{1_a}"', from=1-3, to=3-1]
            \arrow["{\langle1_a, f \rangle}"', from=1-3, to=3-3, shift right=1, dashed]
            \arrow["{\langle1_a, g \rangle}", from=1-3, to=3-3, shift left=1, dashed]
            \arrow["{pr_b}"', from=3-3, to=3-5]
        \end{tikzcd}\]
        Notice that we get the following facts which you can wager will be made use of shortly:
        $$pr_a \o \langle 1_a, f \rangle = 1_a,$$
        $$pr_a \o \langle1_a, g \rangle = 1_a,$$
        $$pr_b \o \langle 1_a, f \rangle = f,$$
        $$pr_b \o \langle1_a, g \rangle = g.$$
        Since $\langle1_a, f \rangle$ and $\langle1_a, g \rangle$ share a codomain we can form the pullback:
        % https://q.uiver.app/?q=WzAsNCxbMiwwLCJhIl0sWzIsMiwiYSBcXHRpbWVzIGIiXSxbMCwyLCJhIl0sWzAsMCwiZCJdLFsyLDEsIlxcbGFuZ2xlIDFfYSwgZiBcXHJhbmdsZSJdLFswLDEsIlxcbGFuZ2xlIDFfYSwgZyBcXHJhbmdsZSJdLFszLDIsInAiLDJdLFszLDAsInEiXV0=
        \[\begin{tikzcd}
            {d} && {a} \\
            \\
            {a} && {a \times b}
            \arrow["{\langle1_a, f \rangle}", from=3-1, to=3-3]
            \arrow["{\langle1_a, g \rangle}", from=1-3, to=3-3]
            \arrow["{p}"', from=1-1, to=3-1]
            \arrow["{q}", from=1-1, to=1-3]
        \end{tikzcd}\]
        with $\langle 1_a, f \rangle \o p = \langle1_a, g \rangle \o q$.
        We can tack on the projection functions to move forward:
        % https://q.uiver.app/?q=WzAsNixbMiwwLCJhIl0sWzIsMiwiYSBcXHRpbWVzIGIiXSxbMCwyLCJhIl0sWzAsMCwiZCJdLFszLDMsImEiXSxbMSwzLCJiIl0sWzIsMSwiXFxsYW5nbGUgMV9hLCBmIFxccmFuZ2xlIl0sWzAsMSwiXFxsYW5nbGUgMV9hLCBnIFxccmFuZ2xlIl0sWzMsMiwicCIsMl0sWzMsMCwicSJdLFsxLDQsInByX2EiLDJdLFsxLDUsInByX2IiXV0=
        \[\begin{tikzcd}
            {d} && {a} \\
            \\
            {a} && {a \times b} \\
            & {b} && {a}
            \arrow["{\langle 1_a, f \rangle}", from=3-1, to=3-3]
            \arrow["{\langle 1_a, g \rangle}", from=1-3, to=3-3]
            \arrow["{p}"', from=1-1, to=3-1]
            \arrow["{q}", from=1-1, to=1-3]
            \arrow["{pr_a}"', from=3-3, to=4-4]
            \arrow["{pr_b}", from=3-3, to=4-2]
        \end{tikzcd}\]
        First, 
        $$pr_a \o \langle 1_a, f \rangle \o p = pr_a \o \langle1_a, g \rangle \o q.$$
        Substituting from above gives
        $$1_a \o p = 1_a \o q,$$
        or $p = q$.
        So we can dispense with $q$ from here on. Let's tack on the other projection function:
        $$pr_b \o \langle 1_a, f \rangle \o p = pr_b \o \langle1_a, g \rangle \o p.$$
        Substituting from above,
        $$f \o p = g \o p.$$
        Therefore $p$ is an equalizer of $f$ and $g$.

    \end{exercise}

    \begin{exercise}
        Working in a category with products and equalizers, we demonstrate how to construct pullbacks.
        % Given $f : a \to c$ and $g : b \to c$ we first form the products $a \times c$ and $b \times c$,
        % and retrieve the product arrows $\langle 1_a, f \rangle : a \to a \times c$ and $\langle 1_b, g \rangle : b \to b \times c$

        % % https://q.uiver.app/?q=WzAsNyxbMCwwLCJhIl0sWzIsMCwiYyJdLFs0LDAsImIiXSxbMSwwLCJhIFxcdGltZXMgYyJdLFszLDAsImIgXFx0aW1lcyBjICJdLFsxLDIsImEiXSxbMywyLCJiIl0sWzMsMF0sWzMsMV0sWzQsMV0sWzQsMl0sWzUsMSwiZiIsMl0sWzYsMSwiZyJdLFs1LDAsIjFfYSJdLFs2LDIsIjFfYiIsMl0sWzUsMywiXFxsYW5nbGUgMV9hLCBmIFxccmFuZ2xlIiwxLHsic3R5bGUiOnsiYm9keSI6eyJuYW1lIjoiZGFzaGVkIn19fV0sWzYsNCwiXFxsYW5nbGUgMV9iLCBnIFxccmFuZ2xlIiwxLHsic3R5bGUiOnsiYm9keSI6eyJuYW1lIjoiZGFzaGVkIn19fV1d
        % \[\begin{tikzcd}
        %     {a} & {a \times c} & {c} & {b \times c } & {b} \\
        %     \\
        %     & {a} && {b}
        %     \arrow[from=1-2, to=1-1]
        %     \arrow[from=1-2, to=1-3]
        %     \arrow[from=1-4, to=1-3]
        %     \arrow[from=1-4, to=1-5]
        %     \arrow["{f}"', from=3-2, to=1-3]
        %     \arrow["{g}", from=3-4, to=1-3]
        %     \arrow["{1_a}", from=3-2, to=1-1]
        %     \arrow["{1_b}"', from=3-4, to=1-5]
        %     \arrow["{\langle 1_a, f \rangle}" description, from=3-2, to=1-2, dashed]
        %     \arrow["{\langle 1_b, g \rangle}" description, from=3-4, to=1-4, dashed]
        % \end{tikzcd}\]
        Given $f : a \to c$ and $g : b \to c$, form the product $a \times b$ and the parallel functions $f \circ pr_a, g \circ pr_b : a \times b \too c$:
        % https://q.uiver.app/?q=WzAsNCxbMCwxLCJhXFx0aW1lcyBiIl0sWzIsMCwiYSJdLFsyLDIsImIiXSxbNCwxLCJjIl0sWzAsMSwicHJfYSJdLFswLDIsInByX2IiLDJdLFsxLDMsImYiXSxbMiwzLCJnIiwyXV0=
        \[\begin{tikzcd}
            && {a} \\
            {a\times b} &&&& {c} \\
            && {b}
            \arrow["{pr_a}", from=2-1, to=1-3]
            \arrow["{pr_b}"', from=2-1, to=3-3]
            \arrow["{f}", from=1-3, to=2-5]
            \arrow["{g}"', from=3-3, to=2-5]
        \end{tikzcd}\]
        Then there is an equalizer $i : e \to a \times b$ such that $(f \o pr_a) \o i = (g \o pr_b) \o i$:
        % https://q.uiver.app/?q=WzAsMyxbMiwwLCJhXFx0aW1lcyBiIl0sWzQsMCwiYyJdLFswLDAsImUiXSxbMCwxLCJmIFxcY2lyYyBwcl9hIiwxLHsiY3VydmUiOi0xfV0sWzAsMSwiZyBcXGNpcmMgcHJfYiIsMSx7ImN1cnZlIjoxfV0sWzIsMCwiaSIsMV1d
        \[\begin{tikzcd}
            {e} && {a\times b} && {c}
            \arrow["{f \circ pr_a}" description, from=1-3, to=1-5, curve={height=-6pt}]
            \arrow["{g \circ pr_b}" description, from=1-3, to=1-5, curve={height=6pt}]
            \arrow["{i}" description, from=1-1, to=1-3]
        \end{tikzcd}\]
        But then $f \o (pr_a \o i) = g \o (pr_b \o i)$ meaning 
        % https://q.uiver.app/?q=WzAsNCxbMCwwLCJlIl0sWzAsMiwiYSJdLFsyLDAsImIiXSxbMiwyLCJjIl0sWzIsMywiZyJdLFsxLDMsImYiLDJdLFswLDIsInByX2IgXFxjaXJjIGkiXSxbMCwxLCJwcl9hIFxcY2lyYyBpIiwyXV0=
        \[\begin{tikzcd}
            {e} && {b} \\
            \\
            {a} && {c}
            \arrow["{g}", from=1-3, to=3-3]
            \arrow["{f}"', from=3-1, to=3-3]
            \arrow["{pr_b \circ i}", from=1-1, to=1-3]
            \arrow["{pr_a \circ i}"', from=1-1, to=3-1]
        \end{tikzcd}\]
        commutes. 
        % If we have another $e', i'$ so that $f \o pr_a \o i' = g \o pr_b \o i'$, then there is a unique $h$ factoring through our original equalizer $e$:
        % % https://q.uiver.app/?q=WzAsNCxbMiwyLCJhXFx0aW1lcyBiIl0sWzQsMiwiYyJdLFswLDIsImUiXSxbMCwwLCJlJyJdLFswLDEsImYgXFxjaXJjIHByX2EiLDEseyJjdXJ2ZSI6LTF9XSxbMCwxLCJnIFxcY2lyYyBwcl9iIiwxLHsiY3VydmUiOjF9XSxbMiwwLCJpIiwxXSxbMywwLCJpJyJdLFszLDIsImgiLDAseyJzdHlsZSI6eyJib2R5Ijp7Im5hbWUiOiJkYXNoZWQifX19XV0=
        % \[\begin{tikzcd}
        %     {e'} \\
        %     \\
        %     {e} && {a\times b} && {c}
        %     \arrow["{f \circ pr_a}" description, from=3-3, to=3-5, curve={height=-6pt}]
        %     \arrow["{g \circ pr_b}" description, from=3-3, to=3-5, curve={height=6pt}]
        %     \arrow["{i}" description, from=3-1, to=3-3]
        %     \arrow["{i'}", from=1-1, to=3-3]
        %     \arrow["{h}", from=1-1, to=3-1, dashed]
        % \end{tikzcd}\]
        To see universality, suppose there is another $e', i_a, i_b$ such that $f\o i_a = g \o i_b$, or  
        % https://q.uiver.app/?q=WzAsNSxbMSwxLCJlIl0sWzEsMywiYSJdLFszLDEsImIiXSxbMywzLCJjIl0sWzAsMCwiZSciXSxbMiwzLCJnIl0sWzEsMywiZiIsMl0sWzAsMiwicHJfYiBcXGNpcmMgaSJdLFswLDEsInByX2EgXFxjaXJjIGkiLDJdLFs0LDIsImlfYiIsMCx7ImN1cnZlIjotM31dLFs0LDEsImlfYSIsMix7ImN1cnZlIjo0fV1d
        \[\begin{tikzcd}
            {e'} \\
            & {e} && {b} \\
            \\
            & {a} && {c}
            \arrow["{g}", from=2-4, to=4-4]
            \arrow["{f}"', from=4-2, to=4-4]
            \arrow["{pr_b \circ i}", from=2-2, to=2-4]
            \arrow["{pr_a \circ i}"', from=2-2, to=4-2]
            \arrow["{i_b}", from=1-1, to=2-4, curve={height=-18pt}]
            \arrow["{i_a}"', from=1-1, to=4-2, curve={height=18pt}]
        \end{tikzcd}\]
        commutes. Then we use the product $a \times b$ to retrieve the unique function $\langle i_a, i_b \rangle$:
        % https://q.uiver.app/?q=WzAsNCxbMSwyLCJhXFx0aW1lcyBiIl0sWzAsMiwiYSJdLFsyLDIsImIiXSxbMSwwLCJlJyJdLFswLDFdLFswLDJdLFszLDEsImlfYSIsMl0sWzMsMiwiaV9iIl0sWzMsMCwiXFxsYW5nbGUgaV9hLCBpX2IgXFxyYW5nbGUiLDEseyJzdHlsZSI6eyJib2R5Ijp7Im5hbWUiOiJkYXNoZWQifX19XV0=
        \[\begin{tikzcd}
            & {e'} \\
            \\
            {a} & {a\times b} & {b}
            \arrow[from=3-2, to=3-1]
            \arrow[from=3-2, to=3-3]
            \arrow["{i_a}"', from=1-2, to=3-1]
            \arrow["{i_b}", from=1-2, to=3-3]
            \arrow["{\langle i_a, i_b \rangle}" description, from=1-2, to=3-2, dashed]
        \end{tikzcd}\]
        Here $i_a = pr_a \o \langle i_a, i_b \rangle$ and $i_b = pr_b \o \langle i_a, i_b \rangle$.
        But since $f\o i_a = g \o i_b$ we have
        $f \o pr_a \o \langle i_a, i_b \rangle = g \o pr_b \o \langle i_a, i_b \rangle$,
        meaning that $\langle i_a, i_b \rangle$ functions as an equalizer:
        % https://q.uiver.app/?q=WzAsNCxbMiwyLCJhXFx0aW1lcyBiIl0sWzQsMiwiYyJdLFswLDIsImUiXSxbMCwwLCJlJyJdLFswLDEsImYgXFxjaXJjIHByX2EiLDEseyJjdXJ2ZSI6LTF9XSxbMCwxLCJnIFxcY2lyYyBwcl9iIiwxLHsiY3VydmUiOjF9XSxbMiwwLCJpIiwxXSxbMywwLCJcXGxhbmdsZSBpX2EsIGlfYiBcXHJhbmdsZSIsMV0sWzMsMiwiaCIsMSx7InN0eWxlIjp7ImJvZHkiOnsibmFtZSI6ImRhc2hlZCJ9fX1dXQ==
        \[\begin{tikzcd}
            {e'} \\
            \\
            {e} && {a\times b} && {c}
            \arrow["{f \circ pr_a}" description, from=3-3, to=3-5, curve={height=-6pt}]
            \arrow["{g \circ pr_b}" description, from=3-3, to=3-5, curve={height=6pt}]
            \arrow["{i}" description, from=3-1, to=3-3]
            \arrow["{\langle i_a, i_b \rangle}" description, from=1-1, to=3-3]
            \arrow["{h}" description, from=1-1, to=3-1, dashed]
        \end{tikzcd}\]
        So we get the unique factorization $h : e' \to e$ from the equalizer's universal property, and plug it straight back into our commutative square to see that we really have the universal pullback:
        % https://q.uiver.app/?q=WzAsNSxbMSwxLCJlIl0sWzEsMywiYSJdLFszLDEsImIiXSxbMywzLCJjIl0sWzAsMCwiZSciXSxbMiwzLCJnIl0sWzEsMywiZiIsMl0sWzAsMiwicHJfYiBcXGNpcmMgaSJdLFswLDEsInByX2EgXFxjaXJjIGkiLDJdLFs0LDIsImlfYiIsMCx7ImN1cnZlIjotMX1dLFs0LDEsImlfYSIsMix7ImN1cnZlIjoyfV0sWzQsMCwiaCIsMCx7InN0eWxlIjp7ImJvZHkiOnsibmFtZSI6ImRhc2hlZCJ9fX1dXQ==
        \[\begin{tikzcd}
            {e'} \\
            & {e} && {b} \\
            \\
            & {a} && {c}
            \arrow["{g}", from=2-4, to=4-4]
            \arrow["{f}"', from=4-2, to=4-4]
            \arrow["{pr_b \circ i}", from=2-2, to=2-4]
            \arrow["{pr_a \circ i}"', from=2-2, to=4-2]
            \arrow["{i_b}", from=1-1, to=2-4, curve={height=-18pt}]
            \arrow["{i_a}"', from=1-1, to=4-2, curve={height=18pt}]
            \arrow["{h}", from=1-1, to=2-2, dashed]
        \end{tikzcd}\]

    \end{exercise}

\section{Exponentials}
    In \textbf{Set}, think of the exponential object $B^A$ as all functions from $A$ to $B$.
    Then we can apply an element $f \in B^A$ to take $a \in A$ to some $b \in B$ (namely $f(a)$).
    In other words we have the evaluation function $ev : B^A \times A \to B$ that takes $\langle f, a \rangle$ to $f(a)$.
    Furthermore, $ev$ has a universal property. 

    The action of any function 
    $g: C \times A \to B$ 
    determines a unique function in $B^A$ - but there is an in-between step.
    Namely, for any particular $c$, we first `curry' $g$ - Leave $c$ fixed and evaluate $g(c, a)$ for every $a \in A$. Then $g$ (\emph{for that $c$ only}) is reduced to a function $g_c : A \to B$, or in other words a member of $B^A$.
    Now we can lift the $g_c$'s to get a function $\hat{g} : C \to B^A$. Simply let 
    $$\hat{g}(c) = g_c.$$
    This is the only choice so that
    $$ev(\langle \hat{g}(c),a \rangle) = g(\langle c, a \rangle)$$
    for any $c \in C$ and $a \in A$.

    I'm sure you are crying out for diagrams by this point, so let's get to the general definition:

    \begin{defi}
        If our category $C$ has \emph{exponentiation} then given any two objects $a$ and $b$,
        there exists an object $b^a$ and an arrow $ev : b^a \times a \to b$,
        so that any $c$ and $g : c \times a \to b$,
        there is a unique $\hat{g} : c \to b^a$:
        % https://q.uiver.app/?q=WzAsNyxbMiwwLCJhIl0sWzQsMSwiYiJdLFswLDAsImJeYSJdLFsxLDAsImJeYSBcXHRpbWVzIGEiXSxbMCwyLCJjIl0sWzEsMiwiY1xcdGltZXMgYSJdLFsyLDIsImEiXSxbMywxLCJldiJdLFszLDJdLFszLDBdLFs1LDEsImciLDJdLFs1LDRdLFs1LDZdLFs0LDIsIlxcaGF0e2d9IiwwLHsic3R5bGUiOnsiYm9keSI6eyJuYW1lIjoiZGFzaGVkIn19fV1d
        \[\begin{tikzcd}
            {b^a} & {b^a \times a} & {a} \\
            &&&& {b} \\
            {c} & {c\times a} & {a}
            \arrow["{ev}", from=1-2, to=2-5]
            \arrow[from=1-2, to=1-1]
            \arrow[from=1-2, to=1-3]
            \arrow["{g}"', from=3-2, to=2-5]
            \arrow[from=3-2, to=3-1]
            \arrow[from=3-2, to=3-3]
            \arrow["{\hat{g}}", from=3-1, to=1-1, dashed]
        \end{tikzcd}\]
        so that when we form the product arrow $\hat{g} \times 1_a$, we have $ev \o (\hat{g} \times 1_a) = g$:
        % https://q.uiver.app/?q=WzAsNyxbMywwLCJhIl0sWzIsMSwiYiJdLFswLDAsImJeYSJdLFsxLDAsImJeYSBcXHRpbWVzIGEiXSxbMCwyLCJjIl0sWzEsMiwiY1xcdGltZXMgYSJdLFszLDIsImEiXSxbMywxLCJldiJdLFszLDJdLFszLDBdLFs1LDEsImciLDJdLFs1LDRdLFs1LDZdLFs0LDIsIlxcaGF0e2d9IiwwLHsic3R5bGUiOnsiYm9keSI6eyJuYW1lIjoiZGFzaGVkIn19fV0sWzYsMCwiMV9hIl0sWzUsMywiXFxoYXR7Z30gXFx0aW1lcyAxX2EiLDAseyJzdHlsZSI6eyJib2R5Ijp7Im5hbWUiOiJkYXNoZWQifX19XV0=
        \[\begin{tikzcd}
            {b^a} & {b^a \times a} && {a} \\
            && {b} \\
            {c} & {c\times a} && {a}
            \arrow["{ev}", from=1-2, to=2-3]
            \arrow[from=1-2, to=1-1]
            \arrow[from=1-2, to=1-4]
            \arrow["{g}"', from=3-2, to=2-3]
            \arrow[from=3-2, to=3-1]
            \arrow[from=3-2, to=3-4]
            \arrow["{\hat{g}}", from=3-1, to=1-1, dashed]
            \arrow["{1_a}", from=3-4, to=1-4]
            \arrow["{\hat{g} \times 1_a}", from=3-2, to=1-2, dashed]
        \end{tikzcd}\]
    \end{defi}

    \begin{defi}
        In the context of the last definition,
        the two arrows $g$ and $\hat{g}$ are \emph{exponential adjoints}. 
    \end{defi}

    \begin{fact} \label{exponential adjoint bijection}
        Furthermore there is a one-to-one correspondence (bijection) between arrows
        $$(c \times a) \to b \quad \text{ and } \quad c \to b^a.$$
        In other words
        $$C(c\times a, b) \cong C(c, b^a).$$
        % https://q.uiver.app/?q=WzAsNCxbMCwwLCJiIl0sWzIsMCwiYl5hIl0sWzIsMiwiYyJdLFswLDIsImNcXHRpbWVzIGEiXSxbMywwLCJnIiwyXSxbMiwxLCJcXGhhdHtnfSJdLFs1LDQsIiIsMCx7Imxlbmd0aCI6NzAsInN0eWxlIjp7InRhaWwiOnsibmFtZSI6ImFycm93aGVhZCJ9fX1dXQ==
        \[\begin{tikzcd}
            {b} && {b^a} \\
            \\
            {c\times a} && {c}
            \arrow["{g}"{name=0, swap}, from=3-1, to=1-1]
            \arrow["{\hat{g}}"{name=1}, from=3-3, to=1-3]
            \arrow[Rightarrow, from=1, to=0, shorten <=10pt, shorten >=10pt, 2tail reversed]
        \end{tikzcd}\]

        Let's examine the claim that we've got a bijection from $C(c\times a, b) \to C(c, b^a)$.
        Given $\hat{g}, \hat{h} : c \to b^a$, if $\hat{g} = \hat{h}$ then we have
        $ev \o (\hat{h}\times 1_a) = ev \o (\hat{g} \times 1_a)$:
        % https://q.uiver.app/?q=WzAsNSxbNCwxLCJiIl0sWzAsMCwiYl5hIl0sWzIsMCwiYl5hIFxcdGltZXMgYSJdLFswLDIsImMiXSxbMiwyLCJjXFx0aW1lcyBhIl0sWzIsMCwiZXYiXSxbMiwxXSxbNCwwLCJnIiwyXSxbNCwzXSxbMywxLCJcXGhhdHtnfSIsMCx7ImN1cnZlIjotMX1dLFs0LDIsIlxcaGF0e2d9IFxcdGltZXMgMV9hIiwwLHsiY3VydmUiOi0xfV0sWzMsMSwiXFxoYXR7aH0iLDAseyJjdXJ2ZSI6MX1dLFs0LDIsIlxcaGF0e2h9IFxcdGltZXMgMV9hIiwyLHsiY3VydmUiOjF9XV0=
        \[\begin{tikzcd}
            {b^a} && {b^a \times a} \\
            &&&& {b} \\
            {c} && {c\times a}
            \arrow["{ev}", from=1-3, to=2-5]
            \arrow[from=1-3, to=1-1]
            \arrow["{g}"', from=3-3, to=2-5]
            \arrow[from=3-3, to=3-1]
            \arrow["{\hat{g}}", from=3-1, to=1-1, curve={height=-6pt}]
            \arrow["{\hat{g} \times 1_a}", from=3-3, to=1-3, curve={height=-6pt}]
            \arrow["{\hat{h}}", from=3-1, to=1-1, curve={height=6pt}]
            \arrow["{\hat{h} \times 1_a}"', from=3-3, to=1-3, curve={height=6pt}]
        \end{tikzcd}\]
        Thus the assignment is injective.

        Now for any given $\hat{g} : c \to b^a$, we can form the arrow $ev \o (\hat{g} \times 1_a) : c \times a \to b$. Let's call it $h$.
        % https://q.uiver.app/?q=WzAsNSxbNCwxLCJiIl0sWzAsMCwiYl5hIl0sWzIsMCwiYl5hIFxcdGltZXMgYSJdLFswLDIsImMiXSxbMiwyLCJjXFx0aW1lcyBhIl0sWzIsMCwiZXYiXSxbMiwxXSxbNCwwLCJoID0gZXYgXFxvIChcXGhhdHtnfSBcXHRpbWVzIDFfYSkiLDJdLFs0LDNdLFszLDEsIlxcaGF0e2d9Il0sWzQsMiwiXFxoYXR7Z30gXFx0aW1lcyAxX2EiXV0=
        \[\begin{tikzcd}
            {b^a} && {b^a \times a} \\
            &&&& {b} \\
            {c} && {c\times a}
            \arrow["{ev}", from=1-3, to=2-5]
            \arrow[from=1-3, to=1-1]
            \arrow["{h = ev \o (\hat{g} \times 1_a)}"', from=3-3, to=2-5]
            \arrow[from=3-3, to=3-1]
            \arrow["{\hat{g}}", from=3-1, to=1-1]
            \arrow["{\hat{g} \times 1_a}", from=3-3, to=1-3]
        \end{tikzcd}\]
        But then from the definition of exponentiation, $h$ determines a \emph{unique} $\hat{h} : c \to b^a$.
        Therefore $\hat{h} = \hat{g}$ and the assignment is surjective.
    \end{fact}
    Let's state a few facts that hold true in a Cartesian Closed category with an initial object $0$ and terminal object $1$.
    \begin{fact} \label{products ignore zeroes up to isomorphism}
        $$0 \cong 0 \times a$$
        for any $a$.
        This follows from applying \cref{exponential adjoint bijection} for any object $b$: 
        $$C(0 , b^a) \cong C(0 \times a, b)$$
        Since $C(0 , b^a)$ has 1 member, so does $C(0 \times a, b)$. But since $b$ was arbitrary,
        $0 \times a$ is initial. Lastly because of \cref{initial objects unique up to isomorphism} we get 
        $0 \cong 0 \times a.$
    \end{fact}
    \begin{fact} \label{arrow into zero means you ARE zero}
        If there exists an arrow $a \to 0$, then $a \cong 0$.

        Take $f : a \to 0$ and form the product arrow $\langle f, 1_a \rangle : a \to 0 \times a$:
        % https://q.uiver.app/?q=WzAsNCxbMiwwLCJhIl0sWzIsMiwiMCBcXHRpbWVzIGEiXSxbMCwyLCIwIl0sWzQsMiwiYSJdLFswLDIsImYiLDJdLFswLDMsIjFfYSJdLFswLDEsIlxcbGFuZ2xlIGYsIDFfYSBcXHJhbmdsZSIsMCx7InN0eWxlIjp7ImJvZHkiOnsibmFtZSI6ImRhc2hlZCJ9fX1dLFsxLDMsInByX2EiLDFdLFsxLDIsInByXzAiLDFdXQ==
        \[\begin{tikzcd}
            && {a} \\
            \\
            {0} && {0 \times a} && {a}
            \arrow["{f}"', from=1-3, to=3-1]
            \arrow["{1_a}", from=1-3, to=3-5]
            \arrow["{\langle f, 1_a \rangle}", from=1-3, to=3-3, dashed]
            \arrow["{pr_a}" description, from=3-3, to=3-5]
            \arrow["{pr_0}" description, from=3-3, to=3-1]
        \end{tikzcd}\]
        By \cref{product has one identity} we see that $\langle f, 1_a \rangle \o pr_a = 1_{0 \times a}$,
        so
        $$0\times a \cong a,$$
        and by \cref{products ignore zeroes up to isomorphism},
        $$a \cong 0.$$
    \end{fact}
    \begin{fact}
        If $0 \cong 1$ then all objects are isomorphic to each other.

        Any object has an arrow to $1$, and since $0 cong 1$ it will have an arrow to $0$.
        Then due to \cref{arrow into zero means you ARE zero}, every object is isomorphic to $0$.
    \end{fact}
    \begin{fact}
        An arrow $f : 0 \to a$ for any $a$ is monic.

        Suppose we have $g, h : b \too 0$ with $f\o g = f \o h$:
        % https://q.uiver.app/?q=WzAsMyxbMiwwLCIwIl0sWzQsMCwiYSJdLFswLDAsImIiXSxbMCwxLCJmIl0sWzIsMCwiZyIsMCx7Im9mZnNldCI6LTF9XSxbMiwwLCJoIiwyLHsib2Zmc2V0IjoxfV1d
        \[\begin{tikzcd}
            {b} && {0} && {a}
            \arrow["{f}", from=1-3, to=1-5]
            \arrow["{g}", from=1-1, to=1-3, shift left=1]
            \arrow["{h}"', from=1-1, to=1-3, shift right=1]
        \end{tikzcd}\]
        By \cref{arrow into zero means you ARE zero}, $b \cong 0$, meaning the arrow from $b \to 0$ is unique and $g = h$.
    \end{fact}
    \begin{exercise}
        \begin{enumerate}
            \item $$a^1 \cong a$$

                \cref{exponential adjoint bijection} gives for $a^1$ and any $c$ that
                $$C(c\times 1, a) \cong C(c, a^1).$$
                In particular, plugging in $a$ gives
                $$C(a\times 1, a) \cong C(a, a^1).$$
                TODO: Why does the bijection preserve isos?

            \item $$a^0 \cong 1$$

                \cref{exponential adjoint bijection} gives for $a^0$ and any $c$ that
                $$C(c\times 0, a) \cong C(c, a^0).$$
                From \cref{products ignore zeroes up to isomorphism} we know that $c \times 0 \cong 0$,
                so $C(c\times 0, a)$ has 1 element, and $C(c, a^0)$ has one element.
                Since $c$ was arbitrary we get that $a^0$ is terminal and therefore isomorphic to $1$.

            \item $$1^a \cong 1$$

                \cref{exponential adjoint bijection} gives for $1^a$ and any $c$ that 
                $$C(c \times a, 1) \cong C(c, 1^a).$$
                Since $1$ is terminal, $C(c \times a, 1)$ has one member. Thus $C(c, 1^a)$ has one member, meaning $1^a$ is terminal and
                $1^a \cong 1$.
        \end{enumerate}

    \end{exercise}