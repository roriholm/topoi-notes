\chapter{Mathematics = Set Theory?}

\section{Where Set Theory Cracks}

If it starts to feel like you're working with a system that's \emph{too} powerful, try this: turn the system on itself.

Maybe you'll feed the system to itself, as in the proof of the undecidability of the Halting Problem. 
Maybe you'll reduce its power to a handful of (say) 3 properties and show that any 2 precludes the possiblity of the 3rd:
\begin{itemize}
  \item the CAP theorems for distributed databases
  \item Kleinberg's impossibility theorem on clustering algorithms
  \item Arrow's impossibility theorem on social choice
\end{itemize}

In this case we simply hold up the mirror to the system. If we can ask for the set of all sets, we can quickly reach a dead end in the form of Russell's paradox.

We have two different trap doors that let us squeeze tighter and keep moving forward.

Zermelo–Fraenkel set theory tightens the standard on set comprehensions. And it makes sense. It's hard to imagine what the first line of `bad' Python would do:
\begin{verbatim}
    bad_set = {a if p(a)}
    good_set = {b for b in source_set if p(b)}
\end{verbatim}

The von Neumann–Bernays–Gödel (NBG) adds to ZF an abstract notion of `class' allowing us to talk about the `class of all sets'. 
Both provide a strong foundation for modern math, but category theory has emerged as a more abstract foundation.
