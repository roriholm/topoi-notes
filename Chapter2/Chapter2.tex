\documentclass[12pt]{letter}
\usepackage{../common}
\begin{document}

\chapter{2}{What Categories Are}



\begin{definition}
The comma-category $C\downarrow a$ is formed from any category $C$ and any $C$-object $a$.
Its objects are all the $C$-arrows with codomain $a$. (IE, $f_1: c_1 \to a$ and $f_2: c_2 \to a$).
Its arrows are all $C$-arrows between the objects' domains, that commute with the `object arrows'.
(IE, $g : c_1 \to c_2$ so that $f_1 = f_2 \circ g$).
So $C \downarrow a$ looks like this:
\[\begin{tikzcd}
    {f_1} && {f_2}
    \arrow["{g}", from=1-1, to=1-3]
\end{tikzcd}\]
and indicates that this diagram commutes in the original category:
\[\begin{tikzcd}
    {c_1} && {c_2} \\
    \\
    & {a}
    \arrow["{f_1}"', from=1-1, to=3-2]
    \arrow["{f_2}", from=1-3, to=3-2]
    \arrow["{g}", from=1-1, to=1-3]
\end{tikzcd}\]

(TODO: Verify category axioms)
\end{definition}

\begin{example}
Take $C$ to be the preorder on natural numbers, and $a$ to be a given number.
For example, let $a = 3$.

\[\begin{tikzcd}
    {1} & {2} & {\textbf{3}} & {4} & {5} & {\ldots}
    \arrow[from=1-1, to=1-2]
    \arrow[from=1-2, to=1-3]
    \arrow[from=1-3, to=1-4]
    \arrow[from=1-4, to=1-5]
    \arrow[from=1-5, to=1-6]
\end{tikzcd}\]

Objects in $C \downarrow 3$ are statements of `n-less-than-3' relationships.
Arrows in $C \downarrow 3$ from `m-less-than-3' to `n-less-than-3' are `m-less-than-m'.

\[\begin{tikzcd}
    {(1 \leq 3)} && {(2 \leq 3)} && {(3 \leq 3)}
    \arrow["{1 \leq 2}"', from=1-1, to=1-3]
    \arrow["{1 \leq 3}", from=1-1, to=1-5, curve={height=-18pt}]
    \arrow["{2 \leq 3}"', from=1-3, to=1-5]
\end{tikzcd}\]
\end{example}


\begin{example}

Recall $\textbf{Matr(k)}$ has the natural numbers $\N$ as objects, and $(n\times m)$ matrices as arrows from $m \to n$.

Then $\textbf{Matr(k)} \downarrow 3$ has as objects all $3\times n$ matrices where $n \in \N$.

Then if object $A$ is a $3\times m$ matrix, and object $C$ is a $3\times n$ matrix,
then an arrow $B : A \to C$ is an $m \times n$ matrix such that $A=BC$.

Thus the situation in $\textbf{Matr(k)} \downarrow 3$ is
\[\begin{tikzcd}
    {A_{3\times m}} && {C_{3\times n}}
    \arrow["{B_{n\times m}}", from=1-1, to=1-3]
\end{tikzcd}\]

indicating that this diagram commutes in $\textbf{Matr(k)}$:
\[\begin{tikzcd}
    {m} && {n} \\
    \\
    & {3}
    \arrow["{A_{3\times m}}"', from=1-1, to=3-2]
    \arrow["{C_{3\times n}}", from=1-3, to=3-2]
    \arrow["{B_{n\times m}}", from=1-1, to=1-3]
\end{tikzcd}\]

\end{example}

\end{document}