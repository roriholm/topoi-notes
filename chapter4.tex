\chapter{Introducing Topoi}
\section{Subobjects}
    \begin{defi}
        A \emph{subobject} of $d$ is an equivalence class of monic arrows into $d$.
    \end{defi}
    We are using a new notion of equivalence. Start with the idea that a subobject is simply a monic arrow into $d$.
    Now write that $f \subseteq g$ \emph{iff} these is an arrow $h : a \to b$ making
    % https://q.uiver.app/?q=WzAsMyxbMiwxLCJkIl0sWzAsMCwiYSJdLFswLDIsImIiXSxbMiwwLCJnIiwyLHsic3R5bGUiOnsidGFpbCI6eyJuYW1lIjoibW9ubyJ9fX1dLFsxLDAsImYiLDAseyJzdHlsZSI6eyJ0YWlsIjp7Im5hbWUiOiJtb25vIn19fV0sWzEsMiwiaCIsMl1d
    \[\begin{tikzcd}
        {a} \\
        && {d} \\
        {b}
        \arrow["{g}"', from=3-1, to=2-3, tail]
        \arrow["{f}", from=1-1, to=2-3, tail]
        \arrow["{h}"', from=1-1, to=3-1]
    \end{tikzcd}\]
    commute.

    Now say that $f \cong g$ \emph{iff} $f \subseteq g$ and $g \subseteq f$.
    % https://q.uiver.app/?q=WzAsMyxbMiwxLCJkIl0sWzAsMCwiYSJdLFswLDIsImIiXSxbMiwwLCJnIiwyLHsic3R5bGUiOnsidGFpbCI6eyJuYW1lIjoibW9ubyJ9fX1dLFsxLDAsImYiLDAseyJzdHlsZSI6eyJ0YWlsIjp7Im5hbWUiOiJtb25vIn19fV0sWzEsMiwiaCIsMix7ImN1cnZlIjoxfV0sWzIsMSwiaSIsMix7ImN1cnZlIjoxfV1d
    \[\begin{tikzcd}
        {a} \\
        && {d} \\
        {b}
        \arrow["{g}"', from=3-1, to=2-3, tail]
        \arrow["{f}", from=1-1, to=2-3, tail]
        \arrow["{i}"', from=1-1, to=3-1, curve={height=6pt}]
        \arrow["{h}"', from=3-1, to=1-1, curve={height=6pt}]
    \end{tikzcd}\]
    Here, $g \o h = f$ and $f \o i = g$
    \begin{exercise} \label{subobject isomorphism symmetric}
        We wish to show that in the above diagram, $a$ and $b$ are isomorphic.
        Substituting each of our `commuting' equations into the other in turn yields
        $$f \o i \o h = f,$$
        $$g \o h \o i = g.$$
        Since both $f$ and $g$ are monic, left-cancel to get
        $$i \o h = 1_a,$$
        $$h \o i = 1_b.$$
        Conclude that $h$ and $i$ are monic and mutual inverses. Thus $a$ and $b$ are isomorphic.
    \end{exercise}

    \begin{exercise}
        Now we wish to show that $\cong$ in this sense is indeed an equivalence relation.
        \begin{enumerate}
            \item \emph{reflexivity: } $f \cong f$ for $f : a \to b$ since there is an arrow $1_a : a \to a$, and the relationship $a \subseteq a$ is reflexive.

            \item \emph{symmetry: } If $f \cong g$ then $f \subseteq g$ and $g \subseteq f$. So by the symmetry of the word `and' we have that $g \cong f$.

            \item \emph{transitivity: } Suppose $f \cong g$ and if $g \cong k$. Then from the first relation we know $f \subseteq g$ and $g \subseteq f$.  From the second we know $g \subseteq k$ and $k \subseteq g$.

            Since $f \subseteq g$ and $g \subseteq k$, $f \subseteq k$.
            And since $k \subseteq g$ and $g \subseteq f$, $k \subseteq f$.

            Since $f \subseteq k$ and $k \subseteq f$, conclude $f \cong k$.
        \end{enumerate}
    \end{exercise}

    \begin{exercise}
        Suppose that $[f] = [f']$ and $[g] = [g']$.

        ($\Rightarrow$) Suppose $f \subseteq g$.
        Since $f \cong f'$, $f' \subseteq f$.
        Then $f' \subseteq g'$
        
        ($\Leftarrow$) Suppose $f' \subseteq g'$.
        Since $f \cong f'$, $f \subseteq f'$
        Then $f \subseteq g$.
        
        Conclude $f \subseteq g$ iff $f' \subseteq g'$. In other words $\subseteq$ is stable under $\cong$.
    \end{exercise}

    \begin{exercise}
        In \textbf{Set}, $Sub(D) \cong P(D)$.

        First, observe that if $f_1, f_2$ are monic arrows (or equivalently, injective set functions) in a subobject $f \in Sub(D)$, they must have identical images.
        For suppose they don't. Then there is no $h$ that can make
        % https://q.uiver.app/?q=WzAsMyxbMiwxLCJEIl0sWzAsMCwiQSJdLFswLDIsIkIiXSxbMSwwLCJmXzEiXSxbMiwwLCJmXzIiLDJdLFsyLDEsImgiXV0=
        \[\begin{tikzcd}
            {A} \\
            && {D} \\
            {B}
            \arrow["{f_1}", from=1-1, to=2-3]
            \arrow["{f_2}"', from=3-1, to=2-3]
            \arrow["{h}", from=3-1, to=1-1]
        \end{tikzcd}\]
        commute.

        Then each $f \in Sub(D)$ determines a subset $C \subseteq D$, under the image of any particular member. For example, in the above diagram we have $f_1(A) = f_2(B)$. 

        On the other hand, given a subset $C \subseteq D$, the inclusion function $i_C : C \to D$ is monic, and therefore $[i]$ is a subobject of $D$.

        Furthermore we have $[i] = [f_1] = [f_2]$. Thus the image function and the inclusion function are inverses \emph{up to isomorphism},
        so
        $$Sub(D) \cong P(D).$$

    \end{exercise}

    \begin{defi}
        The \emph{naming arrow} for $f$:

        % https://q.uiver.app/?q=WzAsNyxbNCw0LCJhIl0sWzIsNCwiMSBcXHRpbWVzIGEiXSxbMiwyLCJiXmEgXFx0aW1lcyBhIl0sWzAsNCwiMSJdLFswLDIsImJeYSJdLFs0LDIsImEiXSxbNCwwLCJiIl0sWzEsMCwicHJfYSJdLFsxLDMsInByXzEiLDJdLFszLDQsIlxcbGNlaWwgZiBcXHJjZWlsIiwyXSxbMSwyLCJcXGxjZWlsIGYgXFxyY2VpbCBcXHRpbWVzIDFfYSIsMV0sWzIsNF0sWzIsNV0sWzIsNiwiZXYiXSxbNSw2LCJmIl0sWzAsNSwiMV9hIl1d
        \[\begin{tikzcd}
            &&&& {b} \\
            \\
            {b^a} && {b^a \times a} && {a} \\
            \\
            {1} && {1 \times a} && {a}
            \arrow["{pr_a}", from=5-3, to=5-5]
            \arrow["{pr_1}"', from=5-3, to=5-1]
            \arrow["{\lceil f \rceil}"', from=5-1, to=3-1]
            \arrow["{\lceil f \rceil \times 1_a}" description, from=5-3, to=3-3]
            \arrow[from=3-3, to=3-1]
            \arrow[from=3-3, to=3-5]
            \arrow["{ev}", from=3-3, to=1-5]
            \arrow["{f}", from=3-5, to=1-5]
            \arrow["{1_a}", from=5-5, to=3-5]
        \end{tikzcd}\]
    \end{defi}

    \begin{exercise}
        For any element $x : 1 \to a$ of $a$,
        $$ev \o \langle \lceil f \rceil, x \rangle = f \o x$$
    \end{exercise}


\section{Classifying Subobjects}
    We are seeking to generalize the situation where a subset $A \subseteq D$ has a characteristic function $\chi_A : D \to \set{0,1}$, where 
        \[ \chi_A(x) =\begin{cases} 
              1 & x \in A \\
              0 & x \notin A.
           \end{cases}
        \]
    In fact the functions from $D \to \set{0,1}$ are in bijection with all possible subsets of $D$.
    Then given an $f : D \to \set{0,1}$ we define $A_f$ as the subset determined by $f$, and then we have
    % https://q.uiver.app/?q=WzAsNCxbMCwwLCJBX2YiXSxbMiwwLCJEIl0sWzAsMiwiMSJdLFsyLDIsIlxcezAsMVxcfSJdLFswLDEsIiIsMCx7InN0eWxlIjp7InRhaWwiOnsibmFtZSI6Imhvb2siLCJzaWRlIjoidG9wIn19fV0sWzAsMiwiISIsMl0sWzEsMywiZiJdLFsyLDMsInRydWUiLDJdXQ==
    \[\begin{tikzcd}
        {A_f} && {D} \\
        \\
        {1} && {\{0,1\}}
        \arrow[from=1-1, to=1-3, hook]
        \arrow["{!}"', from=1-1, to=3-1]
        \arrow["{f}", from=1-3, to=3-3]
        \arrow["{true}"', from=3-1, to=3-3]
    \end{tikzcd}\]
    is a pullback.

    \begin{defi} \label{subobject classifier}
        If $C$ is a category with a terminal object $1$, then a \emph{subobject classifier} consist of an object
        $\Omega$, and an arrow $true : 1 \to \Omega$.

        The $\Omega$ axiom is that for any monic $f : a \to d$ there is a unique arrow $\chi_f : d \to \Omega$ so that
        % https://q.uiver.app/?q=WzAsNCxbMCwwLCJhIl0sWzIsMCwiZCJdLFswLDIsIjEiXSxbMiwyLCJcXE9tZWdhIl0sWzAsMSwiZiJdLFswLDIsIiEiLDJdLFsxLDMsIlxcY2hpX2YiXSxbMiwzLCJ0cnVlIiwyXV0=
        \[\begin{tikzcd}
            {a} && {d} \\
            \\
            {1} && {\Omega}
            \arrow["{f}", from=1-1, to=1-3]
            \arrow["{!}"', from=1-1, to=3-1]
            \arrow["{\chi_f}", from=1-3, to=3-3]
            \arrow["{true}"', from=3-1, to=3-3]
        \end{tikzcd}\]
        is a pullback square.

        The arrow $\chi_f$ is the \emph{characteristic arrow} of $f$.
    \end{defi}

    \begin{defi} \label{true bang}
        For any $a$ there is a unique arrow $!: a \to 1$. The composite $true \o !$ yields an arrow denoted $true_a$:
        % https://q.uiver.app/?q=WzAsMyxbMiwwLCIxIl0sWzIsMiwiXFxPbWVnYSJdLFswLDAsImEiXSxbMCwxLCJ0cnVlIiwyXSxbMiwwLCIhIl0sWzIsMSwidHJ1ZV9hIiwyXV0=
        \[\begin{tikzcd}
            {a} && {1} \\
            \\
            && {\Omega}
            \arrow["{true}"', from=1-3, to=3-3]
            \arrow["{!}", from=1-1, to=1-3]
            \arrow["{true_a}"', from=1-1, to=3-3]
        \end{tikzcd}\]
    \end{defi}

    \begin{exercise}
        Plugging $ true : 1 \to \Omega$ into \cref{subobject classifier} gives:
        % https://q.uiver.app/?q=WzAsNCxbMCwwLCIxIl0sWzIsMCwiXFxPbWVnYSJdLFswLDIsIjEiXSxbMiwyLCJcXE9tZWdhIl0sWzAsMSwidHJ1ZSIsMCx7InN0eWxlIjp7InRhaWwiOnsibmFtZSI6Im1vbm8ifX19XSxbMCwyLCIhIiwyXSxbMSwzLCJcXGNoaV97dHJ1ZX0iXSxbMiwzLCJ0cnVlIiwyXV0=
        \[\begin{tikzcd}
            {1} && {\Omega} \\
            \\
            {1} && {\Omega}
            \arrow["{true}", from=1-1, to=1-3, tail]
            \arrow["{!}"', from=1-1, to=3-1]
            \arrow["{\chi_{true}}", from=1-3, to=3-3]
            \arrow["{true}"', from=3-1, to=3-3]
        \end{tikzcd}\]
        and we see that $!$ must be the identity $1_1 : 1 \to 1$.
        So for the diagram to be a pullback square we must have
        $$\chi_{true} \o true = true \o 1_1 = true$$
        and therefore
        $$\chi_{true} = 1_\Omega.$$
    \end{exercise}

    \begin{exercise}
        Plugging $1_\Omega : \Omega \to \Omega$ into \cref{subobject classifier} gives 
        % https://q.uiver.app/?q=WzAsNCxbMCwwLCJcXE9tZWdhIl0sWzIsMCwiXFxPbWVnYSJdLFswLDIsIjEiXSxbMiwyLCJcXE9tZWdhIl0sWzAsMSwidHJ1ZSIsMCx7InN0eWxlIjp7InRhaWwiOnsibmFtZSI6Im1vbm8ifX19XSxbMCwyLCIhIiwyXSxbMSwzLCJcXGNoaV97MV9cXE9tZWdhfSJdLFsyLDMsInRydWUiLDJdXQ==
        \[\begin{tikzcd}
            {\Omega} && {\Omega} \\
            \\
            {1} && {\Omega}
            \arrow["{true}", from=1-1, to=1-3, tail]
            \arrow["{!}"', from=1-1, to=3-1]
            \arrow["{\chi_{1_\Omega}}", from=1-3, to=3-3]
            \arrow["{true}"', from=3-1, to=3-3]
        \end{tikzcd}\]
        To be a pullback we must have 
        $\chi_{1_\Omega} \o 1_\Omega = true \o !$.
        Or
        $$\chi_{1_\Omega} = true_\Omega$$
        using the notation of \cref{true bang}.
    \end{exercise}

    \begin{exercise}
        For any $f : a \to b$, since $!_a : a \to 1$ is unique:
        % https://q.uiver.app/?q=WzAsMyxbMiwwLCJiIl0sWzEsMiwiMSJdLFswLDAsImEiXSxbMCwxLCIhX2IiXSxbMiwwLCJmIl0sWzIsMSwiIV9hIiwyXV0=
        \[\begin{tikzcd}
            {a} && {b} \\
            \\
            & {1}
            \arrow["{!_b}", from=1-3, to=3-2]
            \arrow["{f}", from=1-1, to=1-3]
            \arrow["{!_a}"', from=1-1, to=3-2]
        \end{tikzcd}\]
        We must have $!_b \o f = !_a$. Thus
        % https://q.uiver.app/?q=WzAsNCxbMiwwLCJiIl0sWzEsMiwiMSJdLFswLDAsImEiXSxbMSwzLCJcXE9tZWdhIl0sWzAsMSwiIV9iIl0sWzIsMCwiZiJdLFsyLDEsIiFfYSIsMl0sWzEsMywidHJ1ZSJdXQ==
        \[\begin{tikzcd}
            {a} && {b} \\
            \\
            & {1} \\
            & {\Omega}
            \arrow["{!_b}", from=1-3, to=3-2]
            \arrow["{f}", from=1-1, to=1-3]
            \arrow["{!_a}"', from=1-1, to=3-2]
            \arrow["{true}", from=3-2, to=4-2]
        \end{tikzcd}\]
        commutes, or  $true \o !_b \o f = true \o !_a$. Therefore using \cref{true bang},
        $$true_b \o f = true_a.$$
    \end{exercise}

\section{Definition of Topos}
    \begin{defi}
        An \emph{elementary topos} is a category which
        \begin{enumerate}
            \item is finitely complete.
            \item is finitely co-complete.
            \item has exponentiation.
            \item has a subobject classifier.
        \end{enumerate}
    \end{defi}

    \begin{fact}
        More succinctly, a topos can be defined as a Cartesian closed category with a subobject classifier.
    \end{fact}

\section{First Examples}

\section{Bundles and Sheaves}

    \begin{exercise}
        We wish to show that
        % https://q.uiver.app/?q=WzAsNixbMCwwLCJBIl0sWzIsMCwiQiJdLFsxLDEsIkkiXSxbMiwyLCIyIFxcdGltZXMgSSJdLFswLDIsIkkiXSxbMywzXSxbMCwxLCJrIiwwLHsic3R5bGUiOnsidGFpbCI6eyJuYW1lIjoiaG9vayIsInNpZGUiOiJ0b3AifX19XSxbMCwyLCJmIiwyXSxbMSwyLCJnIl0sWzEsMywiXFxjaGlfayJdLFs0LDMsIlQiLDJdLFs0LDIsIjFfSSJdLFswLDQsImYiLDJdLFszLDIsInByX0kiLDJdXQ==
        \[\begin{tikzcd}
            {A} && {B} \\
            & {I} \\
            {I} && {2 \times I} \\
            &&& {}
            \arrow["{k}", from=1-1, to=1-3, hook]
            \arrow["{f}"', from=1-1, to=2-2]
            \arrow["{g}", from=1-3, to=2-2]
            \arrow["{\chi_k}", from=1-3, to=3-3]
            \arrow["{T}"', from=3-1, to=3-3]
            \arrow["{1_I}", from=3-1, to=2-2]
            \arrow["{f}"', from=1-1, to=3-1]
            \arrow["{pr_I}"', from=3-3, to=2-2]
        \end{tikzcd}\]
        is a pullback.
        Note that $1_I : I \to I$ is the terminal object in $Bn(I)$, and $f : A \to I$ is the unique arrow from itself (as an object) to the terminal object. So the desired outcome is to show that $\chi_k \o k = T \o f$.

        Recall the definitions of $T : I \to 2 \times I$ and $\chi_k : B \to 2 \times I$ as product arrows:
        $$T = \langle true \ \o \ !_I, 1_I \rangle,$$
        $$\chi_k = \langle \chi_A, g \rangle:$$

        % https://q.uiver.app/?q=WzAsOSxbNCw1XSxbMiwzLCIyIFxcdGltZXMgSSJdLFsyLDEsIkkiXSxbMiw1LCJCIl0sWzQsMywiSSJdLFswLDMsIjIiXSxbMSwyLCIxIl0sWzIsMCwiQSJdLFsyLDYsIkEiXSxbMSw1XSxbMSw0XSxbMiw2LCIhIiwyXSxbNiw1LCJ0cnVlIiwyXSxbMiw0LCIxX0kiXSxbMyw0LCJnIiwyXSxbMyw1LCJcXGNoaV9BIl0sWzMsMSwiXFxjaGlfayIsMSx7InN0eWxlIjp7ImJvZHkiOnsibmFtZSI6ImRhc2hlZCJ9fX1dLFsyLDEsIlQiLDEseyJzdHlsZSI6eyJib2R5Ijp7Im5hbWUiOiJkYXNoZWQifX19XSxbNywyLCJmIl0sWzgsMywiayIsMl1d
        \[\begin{tikzcd}
            && {A} \\
            && {I} \\
            & {1} \\
            {2} && {2 \times I} && {I} \\
            \\
            && {B} && {} \\
            && {A}
            \arrow[from=4-3, to=4-1]
            \arrow[from=4-3, to=4-5]
            \arrow["{!_I}"', from=2-3, to=3-2]
            \arrow["{true}"', from=3-2, to=4-1]
            \arrow["{1_I}", from=2-3, to=4-5]
            \arrow["{g}"', from=6-3, to=4-5]
            \arrow["{\chi_A}", from=6-3, to=4-1]
            \arrow["{\chi_k}" description, from=6-3, to=4-3, dashed]
            \arrow["{T}" description, from=2-3, to=4-3, dashed]
            \arrow["{f}", from=1-3, to=2-3]
            \arrow["{k}"', from=7-3, to=6-3]
        \end{tikzcd}\]
        Because of \cref{composition distributes through product arrows} then we have
        $$T \o f = \langle true \ \o \ !_I \o f, 1_I \o f \rangle,$$
        $$\chi_k \o k = \langle \chi_A \o k, g \o k \rangle.$$

        By virtue of being an inclusion (and an arrow in $Bn(I)$) we have
        % https://q.uiver.app/?q=WzAsMyxbMCwwLCJBIl0sWzIsMCwiQiJdLFsxLDIsIkkiXSxbMCwxLCJrIiwwLHsic3R5bGUiOnsidGFpbCI6eyJuYW1lIjoiaG9vayIsInNpZGUiOiJ0b3AifX19XSxbMSwyLCJnIl0sWzAsMiwiZiIsMl1d
        \[\begin{tikzcd}
            {A} && {B} \\
            \\
            & {I}
            \arrow["{k}", from=1-1, to=1-3, hook]
            \arrow["{g}", from=1-3, to=3-2]
            \arrow["{f}"', from=1-1, to=3-2]
        \end{tikzcd}\]
        commutes, or $f = g\o k$.
        And since
        % https://q.uiver.app/?q=WzAsNSxbMCwwLCJBIl0sWzIsMCwiQiJdLFsyLDIsIjIiXSxbMCwyLCIxIl0sWzMsM10sWzAsMSwiayIsMCx7InN0eWxlIjp7InRhaWwiOnsibmFtZSI6Imhvb2siLCJzaWRlIjoidG9wIn19fV0sWzEsMiwiXFxjaGlfQSJdLFszLDIsInRydWUiLDJdLFswLDMsIiEiLDJdXQ==
        \[\begin{tikzcd}
            {A} && {B} \\
            \\
            {1} && {2} \\
            &&& {}
            \arrow["{k}", from=1-1, to=1-3, hook]
            \arrow["{\chi_A}", from=1-3, to=3-3]
            \arrow["{true}"', from=3-1, to=3-3]
            \arrow["{!_I}"', from=1-1, to=3-1]
        \end{tikzcd}\]
        is a pullback in \textbf{Set} we get that $\chi_A \o k = true \ \o \ !_I$.

        Lastly, observe that
        $$true \o !_I \o f = true \o !_A$$
        since there is only one arrow $A \to 1$:
        % https://q.uiver.app/?q=WzAsNCxbMSwxLCIxIl0sWzIsMCwiQSJdLFsyLDEsIkkiXSxbMCwyLCIyIl0sWzEsMiwiZiJdLFsyLDAsIiFfSSJdLFsxLDAsIiFfQSIsMl0sWzAsMywidHJ1ZSIsMl1d
        \[\begin{tikzcd}
            && {A} \\
            & {1} & {I} \\
            {2}
            \arrow["{f}", from=1-3, to=2-3]
            \arrow["{!_I}", from=2-3, to=2-2]
            \arrow["{!_A}"', from=1-3, to=2-2]
            \arrow["{true}"', from=2-2, to=3-1]
        \end{tikzcd}\]
        Now putting it all together gives
        $$\langle \chi_A \o k, g \o k \rangle = \langle true \ \o \ !_I \o f, 1_I \o f \rangle,$$
        or 
        $$\langle \chi_A , g  \rangle \o k = \langle true \ \o \ !_I, 1_I \rangle \o f,$$
        or 
        $$T\o f = \chi_k \o k,$$
        as was to be shown.

    \end{exercise}